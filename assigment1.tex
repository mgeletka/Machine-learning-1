% Options for packages loaded elsewhere
\PassOptionsToPackage{unicode}{hyperref}
\PassOptionsToPackage{hyphens}{url}
%
\documentclass[
]{article}
\usepackage{lmodern}
\usepackage{amssymb,amsmath}
\usepackage{ifxetex,ifluatex}
\ifnum 0\ifxetex 1\fi\ifluatex 1\fi=0 % if pdftex
  \usepackage[T1]{fontenc}
  \usepackage[utf8]{inputenc}
  \usepackage{textcomp} % provide euro and other symbols
\else % if luatex or xetex
  \usepackage{unicode-math}
  \defaultfontfeatures{Scale=MatchLowercase}
  \defaultfontfeatures[\rmfamily]{Ligatures=TeX,Scale=1}
\fi
% Use upquote if available, for straight quotes in verbatim environments
\IfFileExists{upquote.sty}{\usepackage{upquote}}{}
\IfFileExists{microtype.sty}{% use microtype if available
  \usepackage[]{microtype}
  \UseMicrotypeSet[protrusion]{basicmath} % disable protrusion for tt fonts
}{}
\makeatletter
\@ifundefined{KOMAClassName}{% if non-KOMA class
  \IfFileExists{parskip.sty}{%
    \usepackage{parskip}
  }{% else
    \setlength{\parindent}{0pt}
    \setlength{\parskip}{6pt plus 2pt minus 1pt}}
}{% if KOMA class
  \KOMAoptions{parskip=half}}
\makeatother
\usepackage{xcolor}
\IfFileExists{xurl.sty}{\usepackage{xurl}}{} % add URL line breaks if available
\IfFileExists{bookmark.sty}{\usepackage{bookmark}}{\usepackage{hyperref}}
\hypersetup{
  pdftitle={Machine learning 1 assigment},
  hidelinks,
  pdfcreator={LaTeX via pandoc}}
\urlstyle{same} % disable monospaced font for URLs
\usepackage[margin=1in]{geometry}
\usepackage{color}
\usepackage{fancyvrb}
\newcommand{\VerbBar}{|}
\newcommand{\VERB}{\Verb[commandchars=\\\{\}]}
\DefineVerbatimEnvironment{Highlighting}{Verbatim}{commandchars=\\\{\}}
% Add ',fontsize=\small' for more characters per line
\usepackage{framed}
\definecolor{shadecolor}{RGB}{248,248,248}
\newenvironment{Shaded}{\begin{snugshade}}{\end{snugshade}}
\newcommand{\AlertTok}[1]{\textcolor[rgb]{0.94,0.16,0.16}{#1}}
\newcommand{\AnnotationTok}[1]{\textcolor[rgb]{0.56,0.35,0.01}{\textbf{\textit{#1}}}}
\newcommand{\AttributeTok}[1]{\textcolor[rgb]{0.77,0.63,0.00}{#1}}
\newcommand{\BaseNTok}[1]{\textcolor[rgb]{0.00,0.00,0.81}{#1}}
\newcommand{\BuiltInTok}[1]{#1}
\newcommand{\CharTok}[1]{\textcolor[rgb]{0.31,0.60,0.02}{#1}}
\newcommand{\CommentTok}[1]{\textcolor[rgb]{0.56,0.35,0.01}{\textit{#1}}}
\newcommand{\CommentVarTok}[1]{\textcolor[rgb]{0.56,0.35,0.01}{\textbf{\textit{#1}}}}
\newcommand{\ConstantTok}[1]{\textcolor[rgb]{0.00,0.00,0.00}{#1}}
\newcommand{\ControlFlowTok}[1]{\textcolor[rgb]{0.13,0.29,0.53}{\textbf{#1}}}
\newcommand{\DataTypeTok}[1]{\textcolor[rgb]{0.13,0.29,0.53}{#1}}
\newcommand{\DecValTok}[1]{\textcolor[rgb]{0.00,0.00,0.81}{#1}}
\newcommand{\DocumentationTok}[1]{\textcolor[rgb]{0.56,0.35,0.01}{\textbf{\textit{#1}}}}
\newcommand{\ErrorTok}[1]{\textcolor[rgb]{0.64,0.00,0.00}{\textbf{#1}}}
\newcommand{\ExtensionTok}[1]{#1}
\newcommand{\FloatTok}[1]{\textcolor[rgb]{0.00,0.00,0.81}{#1}}
\newcommand{\FunctionTok}[1]{\textcolor[rgb]{0.00,0.00,0.00}{#1}}
\newcommand{\ImportTok}[1]{#1}
\newcommand{\InformationTok}[1]{\textcolor[rgb]{0.56,0.35,0.01}{\textbf{\textit{#1}}}}
\newcommand{\KeywordTok}[1]{\textcolor[rgb]{0.13,0.29,0.53}{\textbf{#1}}}
\newcommand{\NormalTok}[1]{#1}
\newcommand{\OperatorTok}[1]{\textcolor[rgb]{0.81,0.36,0.00}{\textbf{#1}}}
\newcommand{\OtherTok}[1]{\textcolor[rgb]{0.56,0.35,0.01}{#1}}
\newcommand{\PreprocessorTok}[1]{\textcolor[rgb]{0.56,0.35,0.01}{\textit{#1}}}
\newcommand{\RegionMarkerTok}[1]{#1}
\newcommand{\SpecialCharTok}[1]{\textcolor[rgb]{0.00,0.00,0.00}{#1}}
\newcommand{\SpecialStringTok}[1]{\textcolor[rgb]{0.31,0.60,0.02}{#1}}
\newcommand{\StringTok}[1]{\textcolor[rgb]{0.31,0.60,0.02}{#1}}
\newcommand{\VariableTok}[1]{\textcolor[rgb]{0.00,0.00,0.00}{#1}}
\newcommand{\VerbatimStringTok}[1]{\textcolor[rgb]{0.31,0.60,0.02}{#1}}
\newcommand{\WarningTok}[1]{\textcolor[rgb]{0.56,0.35,0.01}{\textbf{\textit{#1}}}}
\usepackage{graphicx,grffile}
\makeatletter
\def\maxwidth{\ifdim\Gin@nat@width>\linewidth\linewidth\else\Gin@nat@width\fi}
\def\maxheight{\ifdim\Gin@nat@height>\textheight\textheight\else\Gin@nat@height\fi}
\makeatother
% Scale images if necessary, so that they will not overflow the page
% margins by default, and it is still possible to overwrite the defaults
% using explicit options in \includegraphics[width, height, ...]{}
\setkeys{Gin}{width=\maxwidth,height=\maxheight,keepaspectratio}
% Set default figure placement to htbp
\makeatletter
\def\fps@figure{htbp}
\makeatother
\setlength{\emergencystretch}{3em} % prevent overfull lines
\providecommand{\tightlist}{%
  \setlength{\itemsep}{0pt}\setlength{\parskip}{0pt}}
\setcounter{secnumdepth}{-\maxdimen} % remove section numbering

\title{Machine learning 1 assigment}
\author{}
\date{\vspace{-2.5em}}

\begin{document}
\maketitle

\#\#Introduction

The purpose of this assignment is to compare and study different methods
and approaches in the classification of data. We want to focus on how
different methods and their tuning parameters affect the expressiveness
of the final model.

We chose four different methods:

\begin{itemize}
\tightlist
\item
  the knn,
\item
  the linear regression,
\item
  decision trees
\item
  logistic regression.
\end{itemize}

For each method, we plot the error on the test and train dataset and
evaluate the expressiveness of the models with different tuning
parameters. For the evaluation of the methods, we used the Pima Indians
Diabetes available in the R data utils.

\#\#\#\#Importing libraries and loading the data

\begin{Shaded}
\begin{Highlighting}[]
\KeywordTok{library}\NormalTok{(}\StringTok{"mlbench"}\NormalTok{)}
\end{Highlighting}
\end{Shaded}

\begin{verbatim}
## Warning: package 'mlbench' was built under R version 3.6.3
\end{verbatim}

\begin{Shaded}
\begin{Highlighting}[]
\KeywordTok{library}\NormalTok{(}\StringTok{"caTools"}\NormalTok{)}
\end{Highlighting}
\end{Shaded}

\begin{verbatim}
## Warning: package 'caTools' was built under R version 3.6.3
\end{verbatim}

\begin{Shaded}
\begin{Highlighting}[]
\KeywordTok{library}\NormalTok{(}\StringTok{"class"}\NormalTok{)}
\end{Highlighting}
\end{Shaded}

\begin{verbatim}
## Warning: package 'class' was built under R version 3.6.3
\end{verbatim}

\begin{Shaded}
\begin{Highlighting}[]
\KeywordTok{library}\NormalTok{(mice)}
\end{Highlighting}
\end{Shaded}

\begin{verbatim}
## Loading required package: lattice
\end{verbatim}

\begin{verbatim}
## 
## Attaching package: 'mice'
\end{verbatim}

\begin{verbatim}
## The following objects are masked from 'package:base':
## 
##     cbind, rbind
\end{verbatim}

\begin{Shaded}
\begin{Highlighting}[]
\KeywordTok{library}\NormalTok{(grid)}
\KeywordTok{library}\NormalTok{(}\StringTok{"rpart"}\NormalTok{)}
\KeywordTok{library}\NormalTok{(GGally)}
\end{Highlighting}
\end{Shaded}

\begin{verbatim}
## Warning: package 'GGally' was built under R version 3.6.3
\end{verbatim}

\begin{verbatim}
## Loading required package: ggplot2
\end{verbatim}

\begin{verbatim}
## Warning: package 'ggplot2' was built under R version 3.6.2
\end{verbatim}

\begin{verbatim}
## Registered S3 method overwritten by 'GGally':
##   method from   
##   +.gg   ggplot2
\end{verbatim}

\begin{Shaded}
\begin{Highlighting}[]
\NormalTok{utils}\OperatorTok{::}\KeywordTok{data}\NormalTok{(}\StringTok{"PimaIndiansDiabetes2"}\NormalTok{)}
\NormalTok{data <-}\StringTok{ }\NormalTok{PimaIndiansDiabetes2}
\end{Highlighting}
\end{Shaded}

\#\#\#\#Replacing NA values After loading the data, we found out that
some columns contain the NA values. Firstly we deleted all rows with
this value, but these rows were too high. So secondly, we tried to
replace the missing values with their averages, but this introduces much
noise to the data. So finally, we used the mice library and its
imputations methods for dealing with NA values.

\begin{Shaded}
\begin{Highlighting}[]
\KeywordTok{set.seed}\NormalTok{(}\DecValTok{101}\NormalTok{) }
\NormalTok{init =}\StringTok{ }\KeywordTok{mice}\NormalTok{(data, }\DataTypeTok{maxit=}\DecValTok{0}\NormalTok{) }
\NormalTok{data <-}\StringTok{ }\KeywordTok{complete}\NormalTok{(}\KeywordTok{mice}\NormalTok{(data, }\DataTypeTok{method=}\NormalTok{init}\OperatorTok{$}\NormalTok{method, }\DataTypeTok{predictorMatrix=}\NormalTok{init}\OperatorTok{$}\NormalTok{predictorMatrix, }\DataTypeTok{m=}\DecValTok{5}\NormalTok{))}
\end{Highlighting}
\end{Shaded}

\begin{verbatim}
## 
##  iter imp variable
##   1   1  glucose  pressure  triceps  insulin  mass
##   1   2  glucose  pressure  triceps  insulin  mass
##   1   3  glucose  pressure  triceps  insulin  mass
##   1   4  glucose  pressure  triceps  insulin  mass
##   1   5  glucose  pressure  triceps  insulin  mass
##   2   1  glucose  pressure  triceps  insulin  mass
##   2   2  glucose  pressure  triceps  insulin  mass
##   2   3  glucose  pressure  triceps  insulin  mass
##   2   4  glucose  pressure  triceps  insulin  mass
##   2   5  glucose  pressure  triceps  insulin  mass
##   3   1  glucose  pressure  triceps  insulin  mass
##   3   2  glucose  pressure  triceps  insulin  mass
##   3   3  glucose  pressure  triceps  insulin  mass
##   3   4  glucose  pressure  triceps  insulin  mass
##   3   5  glucose  pressure  triceps  insulin  mass
##   4   1  glucose  pressure  triceps  insulin  mass
##   4   2  glucose  pressure  triceps  insulin  mass
##   4   3  glucose  pressure  triceps  insulin  mass
##   4   4  glucose  pressure  triceps  insulin  mass
##   4   5  glucose  pressure  triceps  insulin  mass
##   5   1  glucose  pressure  triceps  insulin  mass
##   5   2  glucose  pressure  triceps  insulin  mass
##   5   3  glucose  pressure  triceps  insulin  mass
##   5   4  glucose  pressure  triceps  insulin  mass
##   5   5  glucose  pressure  triceps  insulin  mass
\end{verbatim}

\hypertarget{summary-of-data}{%
\paragraph{Summary of data}\label{summary-of-data}}

\begin{Shaded}
\begin{Highlighting}[]
\NormalTok{p <-}\StringTok{ }\KeywordTok{ggpairs}\NormalTok{(data[}\KeywordTok{sample}\NormalTok{(}\DecValTok{1}\OperatorTok{:}\KeywordTok{nrow}\NormalTok{(data),}\DecValTok{100}\NormalTok{),], }\DataTypeTok{columns =} \DecValTok{1}\OperatorTok{:}\DecValTok{8}\NormalTok{, ggplot2}\OperatorTok{::}\KeywordTok{aes}\NormalTok{(}\DataTypeTok{colour=}\NormalTok{diabetes), }\DataTypeTok{lower =} \KeywordTok{list}\NormalTok{(}\DataTypeTok{continuous=}\KeywordTok{wrap}\NormalTok{(}\StringTok{"points"}\NormalTok{)), }\DataTypeTok{progress=}\OtherTok{FALSE}\NormalTok{,}\DataTypeTok{legend=}\DecValTok{1}\NormalTok{)}
\NormalTok{p}
\end{Highlighting}
\end{Shaded}

\includegraphics{assigment1_files/figure-latex/unnamed-chunk-3-1.pdf}

\#\#\#\#Dividing the data into train and test dataset

\begin{Shaded}
\begin{Highlighting}[]
\NormalTok{sample <-}\StringTok{ }\KeywordTok{sample.split}\NormalTok{(data, }\DataTypeTok{SplitRatio =} \FloatTok{0.8}\NormalTok{)}
\NormalTok{train <-}\StringTok{ }\KeywordTok{subset}\NormalTok{(data, sample }\OperatorTok{==}\StringTok{ }\OtherTok{TRUE}\NormalTok{)}
\NormalTok{test  <-}\StringTok{ }\KeywordTok{subset}\NormalTok{(data, sample }\OperatorTok{==}\StringTok{ }\OtherTok{FALSE}\NormalTok{)}
\NormalTok{trainDf <-}\StringTok{ }\NormalTok{train[, }\DecValTok{-9}\NormalTok{]}
\NormalTok{testDf <-}\StringTok{ }\NormalTok{test[,}\OperatorTok{-}\DecValTok{9}\NormalTok{]}
\end{Highlighting}
\end{Shaded}

\begin{Shaded}
\begin{Highlighting}[]
\CommentTok{#NA2mean <- function(x) replace(x, is.na(x), mean(x, na.rm = TRUE))}
\CommentTok{#trainDf <- replace(trainDf, TRUE, lapply(trainDf, NA2mean))}
\CommentTok{#testDf <- replace(testDf, TRUE, lapply(testDf, NA2mean))}
\CommentTok{#pred.classes <- knn(trainDf, testDf, train$diabetes, k=5)}
\CommentTok{#testDf <- cbind(testDf,pred.classes)}
\end{Highlighting}
\end{Shaded}

\hypertarget{knn-evaluation}{%
\subsection{KNN evaluation}\label{knn-evaluation}}

We start with the knn method, whose parameter k represents the number of
the neighbors. We plotted different evaluation metrics (accuracy,
precision, and recall) for the test and train data for different values
of the parameter k.

\begin{Shaded}
\begin{Highlighting}[]
\NormalTok{ks <-}\StringTok{ }\KeywordTok{c}\NormalTok{(}\DecValTok{1}\NormalTok{,}\DecValTok{3}\NormalTok{,}\DecValTok{5}\NormalTok{,}\DecValTok{7}\NormalTok{,}\DecValTok{9}\NormalTok{,}\DecValTok{11}\NormalTok{,}\DecValTok{15}\NormalTok{,}\DecValTok{17}\NormalTok{,}\DecValTok{23}\NormalTok{,}\DecValTok{25}\NormalTok{,}\DecValTok{35}\NormalTok{,}\DecValTok{45}\NormalTok{,}\DecValTok{55}\NormalTok{)}\CommentTok{# nearest neighbours to try}
\NormalTok{nks <-}\StringTok{ }\KeywordTok{length}\NormalTok{(ks)}
\NormalTok{accuracy.train <-}\StringTok{ }\KeywordTok{numeric}\NormalTok{(}\DataTypeTok{length=}\NormalTok{nks)}
\NormalTok{accuracy.test <-}\StringTok{ }\KeywordTok{numeric}\NormalTok{(}\DataTypeTok{length=}\NormalTok{nks)}
\NormalTok{precision.train <-}\StringTok{ }\KeywordTok{numeric}\NormalTok{(}\DataTypeTok{length=}\NormalTok{nks)}
\NormalTok{precision.test <-}\StringTok{ }\KeywordTok{numeric}\NormalTok{(}\DataTypeTok{length=}\NormalTok{nks)}
\NormalTok{recall.train <-}\StringTok{ }\KeywordTok{numeric}\NormalTok{(}\DataTypeTok{length=}\NormalTok{nks)}
\NormalTok{recall.test <-}\StringTok{ }\KeywordTok{numeric}\NormalTok{(}\DataTypeTok{length=}\NormalTok{nks)}
\end{Highlighting}
\end{Shaded}

\begin{Shaded}
\begin{Highlighting}[]
\ControlFlowTok{for}\NormalTok{ (i }\ControlFlowTok{in} \KeywordTok{seq}\NormalTok{(}\DataTypeTok{along=}\NormalTok{ks)) \{}
\NormalTok{mod.train <-}\StringTok{ }\KeywordTok{knn}\NormalTok{(trainDf,trainDf,}\DataTypeTok{k=}\NormalTok{ks[i],}\DataTypeTok{cl=}\NormalTok{train}\OperatorTok{$}\NormalTok{diabetes)}
\NormalTok{confussionMatrixTrain =}\StringTok{ }\KeywordTok{table}\NormalTok{(mod.train, train}\OperatorTok{$}\NormalTok{diabetes)}
\NormalTok{accuracy.train[i] =}\StringTok{ }\KeywordTok{sum}\NormalTok{(mod.train }\OperatorTok{==}\StringTok{ }\NormalTok{train}\OperatorTok{$}\NormalTok{diabetes)}\OperatorTok{/}\KeywordTok{length}\NormalTok{(train}\OperatorTok{$}\NormalTok{diabetes)}
\NormalTok{precision.train[i] =}\StringTok{ }\NormalTok{confussionMatrixTrain[}\DecValTok{1}\NormalTok{,}\DecValTok{1}\NormalTok{]}\OperatorTok{/}\KeywordTok{sum}\NormalTok{(confussionMatrixTrain[,}\DecValTok{1}\NormalTok{])}
\NormalTok{recall.train[i] =}\StringTok{ }\NormalTok{confussionMatrixTrain[}\DecValTok{1}\NormalTok{,}\DecValTok{1}\NormalTok{]}\OperatorTok{/}\KeywordTok{sum}\NormalTok{(confussionMatrixTrain[}\DecValTok{1}\NormalTok{,])}
\NormalTok{mod.test <-}\StringTok{ }\KeywordTok{knn}\NormalTok{(trainDf, testDf[,}\OperatorTok{-}\DecValTok{9}\NormalTok{],}\DataTypeTok{k=}\NormalTok{ ks[i],}\DataTypeTok{cl=}\NormalTok{ train}\OperatorTok{$}\NormalTok{diabetes)}
\NormalTok{confussionMatrixTest =}\StringTok{ }\KeywordTok{table}\NormalTok{(mod.test, test}\OperatorTok{$}\NormalTok{diabetes)}
\NormalTok{accuracy.test[i] =}\StringTok{ }\KeywordTok{sum}\NormalTok{(mod.test }\OperatorTok{==}\StringTok{ }\NormalTok{test}\OperatorTok{$}\NormalTok{diabetes)}\OperatorTok{/}\KeywordTok{length}\NormalTok{(test}\OperatorTok{$}\NormalTok{diabetes)}
\NormalTok{precision.test[i] =}\StringTok{ }\NormalTok{confussionMatrixTest[}\DecValTok{1}\NormalTok{,}\DecValTok{1}\NormalTok{]}\OperatorTok{/}\KeywordTok{sum}\NormalTok{(confussionMatrixTest[,}\DecValTok{1}\NormalTok{])}
\NormalTok{recall.test[i] =}\StringTok{ }\NormalTok{confussionMatrixTest[}\DecValTok{1}\NormalTok{,}\DecValTok{1}\NormalTok{]}\OperatorTok{/}\KeywordTok{sum}\NormalTok{(confussionMatrixTest[}\DecValTok{1}\NormalTok{,])}
\NormalTok{\}}
\end{Highlighting}
\end{Shaded}

\hypertarget{section}{%
\subsubsection{}\label{section}}

When we compare the train and the test line in the plot, we see that all
evaluation metrics are in the train line at the maxim with k equals 1,
because we return the real value.

\hypertarget{accuracy}{%
\paragraph{Accuracy}\label{accuracy}}

The accuracy achieves its maximum on the test dataset with k equals 15.

\begin{Shaded}
\begin{Highlighting}[]
\KeywordTok{plot}\NormalTok{(accuracy.train,}\DataTypeTok{xlab=}\StringTok{"Number of NN"}\NormalTok{,}\DataTypeTok{ylab=}\StringTok{"Test error"}\NormalTok{,}\DataTypeTok{type=}\StringTok{"n"}\NormalTok{,}\DataTypeTok{xaxt=}\StringTok{"n"}\NormalTok{, }\DataTypeTok{ylim=}\KeywordTok{c}\NormalTok{(}\FloatTok{0.5}\NormalTok{, }\DecValTok{1}\NormalTok{),  }\DataTypeTok{main=}\StringTok{"Accuracy"}\NormalTok{)}
\KeywordTok{axis}\NormalTok{(}\DecValTok{1}\NormalTok{, }\DecValTok{1}\OperatorTok{:}\KeywordTok{length}\NormalTok{(ks), }\KeywordTok{as.character}\NormalTok{(ks))}
\KeywordTok{lines}\NormalTok{(accuracy.train,}\DataTypeTok{type=}\StringTok{"b"}\NormalTok{,}\DataTypeTok{col=}\StringTok{'red'}\NormalTok{,}\DataTypeTok{pch=}\DecValTok{20}\NormalTok{)}
\KeywordTok{lines}\NormalTok{(accuracy.test,}\DataTypeTok{type=}\StringTok{"b"}\NormalTok{,}\DataTypeTok{col=}\StringTok{'blue'}\NormalTok{,}\DataTypeTok{pch=}\DecValTok{20}\NormalTok{)}
\KeywordTok{legend}\NormalTok{(}\StringTok{"bottomright"}\NormalTok{,}\DataTypeTok{lty=}\DecValTok{1}\NormalTok{,}\DataTypeTok{col=}\KeywordTok{c}\NormalTok{(}\StringTok{"red"}\NormalTok{,}\StringTok{"blue"}\NormalTok{),}\DataTypeTok{legend =} \KeywordTok{c}\NormalTok{(}\StringTok{"train "}\NormalTok{, }\StringTok{"test "}\NormalTok{))}
\end{Highlighting}
\end{Shaded}

\includegraphics{assigment1_files/figure-latex/unnamed-chunk-8-1.pdf}

\hypertarget{precision}{%
\paragraph{Precision}\label{precision}}

The precision achieves its maximum on the test dataset with k equals 23.

\begin{Shaded}
\begin{Highlighting}[]
\KeywordTok{plot}\NormalTok{(precision.train,}\DataTypeTok{xlab=}\StringTok{"Number of NN"}\NormalTok{,}\DataTypeTok{ylab=}\StringTok{"Test error"}\NormalTok{,}\DataTypeTok{type=}\StringTok{"n"}\NormalTok{,}\DataTypeTok{xaxt=}\StringTok{"n"}\NormalTok{, }\DataTypeTok{ylim=}\KeywordTok{c}\NormalTok{(}\FloatTok{0.5}\NormalTok{, }\DecValTok{1}\NormalTok{),  }\DataTypeTok{main=}\StringTok{"Precision"}\NormalTok{)}
\KeywordTok{axis}\NormalTok{(}\DecValTok{1}\NormalTok{, }\DecValTok{1}\OperatorTok{:}\KeywordTok{length}\NormalTok{(ks), }\KeywordTok{as.character}\NormalTok{(ks))}
\KeywordTok{lines}\NormalTok{(precision.train,}\DataTypeTok{type=}\StringTok{"b"}\NormalTok{,}\DataTypeTok{col=}\StringTok{'red'}\NormalTok{,}\DataTypeTok{pch=}\DecValTok{20}\NormalTok{)}
\KeywordTok{lines}\NormalTok{(precision.test,}\DataTypeTok{type=}\StringTok{"b"}\NormalTok{,}\DataTypeTok{col=}\StringTok{'blue'}\NormalTok{,}\DataTypeTok{pch=}\DecValTok{20}\NormalTok{)}
\KeywordTok{legend}\NormalTok{(}\StringTok{"bottomright"}\NormalTok{,}\DataTypeTok{lty=}\DecValTok{1}\NormalTok{,}\DataTypeTok{col=}\KeywordTok{c}\NormalTok{(}\StringTok{"red"}\NormalTok{,}\StringTok{"blue"}\NormalTok{),}\DataTypeTok{legend =} \KeywordTok{c}\NormalTok{(}\StringTok{"train "}\NormalTok{, }\StringTok{"test "}\NormalTok{))}
\end{Highlighting}
\end{Shaded}

\includegraphics{assigment1_files/figure-latex/unnamed-chunk-9-1.pdf}

\hypertarget{recall}{%
\paragraph{Recall}\label{recall}}

The recall achieves its maximum on the test dataset with k equals 15.

\begin{Shaded}
\begin{Highlighting}[]
\KeywordTok{plot}\NormalTok{(recall.train,}\DataTypeTok{xlab=}\StringTok{"Number of NN"}\NormalTok{,}\DataTypeTok{ylab=}\StringTok{"Test error"}\NormalTok{,}\DataTypeTok{type=}\StringTok{"n"}\NormalTok{,}\DataTypeTok{xaxt=}\StringTok{"n"}\NormalTok{, }\DataTypeTok{ylim=}\KeywordTok{c}\NormalTok{(}\FloatTok{0.5}\NormalTok{, }\DecValTok{1}\NormalTok{),  }\DataTypeTok{main=}\StringTok{"Recall"}\NormalTok{)}
\KeywordTok{axis}\NormalTok{(}\DecValTok{1}\NormalTok{, }\DecValTok{1}\OperatorTok{:}\KeywordTok{length}\NormalTok{(ks), }\KeywordTok{as.character}\NormalTok{(ks))}
\KeywordTok{lines}\NormalTok{(recall.train,}\DataTypeTok{type=}\StringTok{"b"}\NormalTok{,}\DataTypeTok{col=}\StringTok{'red'}\NormalTok{,}\DataTypeTok{pch=}\DecValTok{20}\NormalTok{)}
\KeywordTok{lines}\NormalTok{(recall.test,}\DataTypeTok{type=}\StringTok{"b"}\NormalTok{,}\DataTypeTok{col=}\StringTok{'blue'}\NormalTok{,}\DataTypeTok{pch=}\DecValTok{20}\NormalTok{)}
\KeywordTok{legend}\NormalTok{(}\StringTok{"bottomright"}\NormalTok{,}\DataTypeTok{lty=}\DecValTok{1}\NormalTok{,}\DataTypeTok{col=}\KeywordTok{c}\NormalTok{(}\StringTok{"red"}\NormalTok{,}\StringTok{"blue"}\NormalTok{),}\DataTypeTok{legend =} \KeywordTok{c}\NormalTok{(}\StringTok{"train "}\NormalTok{, }\StringTok{"test "}\NormalTok{))}
\end{Highlighting}
\end{Shaded}

\includegraphics{assigment1_files/figure-latex/unnamed-chunk-10-1.pdf}

\#\#\#Plotting the boundaries in different dimensions We implemented the
function plot.knn to visualize the boundaries in the given columns of
indexA and indexB, and then show these two dimensions.

\begin{Shaded}
\begin{Highlighting}[]
\NormalTok{col1<-}\KeywordTok{c}\NormalTok{(}\StringTok{'blue'}\NormalTok{, }\StringTok{'magenta'}\NormalTok{)}
\NormalTok{plot.knn <-}\StringTok{ }\ControlFlowTok{function}\NormalTok{(k, indexA, indexB) \{}
\NormalTok{  grid.A <-}\StringTok{ }\KeywordTok{seq}\NormalTok{(}\KeywordTok{min}\NormalTok{(data[,indexA]), }\KeywordTok{max}\NormalTok{(data[,indexA]), (}\KeywordTok{max}\NormalTok{(data[,indexA]) }\OperatorTok{-}\StringTok{ }\KeywordTok{min}\NormalTok{(data[,indexA])) }\OperatorTok{/}\StringTok{ }\DecValTok{100}\NormalTok{)}
\NormalTok{  grid.B <-}\StringTok{ }\KeywordTok{seq}\NormalTok{(}\KeywordTok{min}\NormalTok{(data[,indexB]), }\KeywordTok{max}\NormalTok{(data[,indexB]), (}\KeywordTok{max}\NormalTok{(data[,indexB]) }\OperatorTok{-}\StringTok{ }\KeywordTok{min}\NormalTok{(data[,indexB])) }\OperatorTok{/}\StringTok{ }\DecValTok{100}\NormalTok{)}
\NormalTok{  grid <-}\StringTok{ }\KeywordTok{expand.grid}\NormalTok{(grid.A,grid.B)}
  \KeywordTok{colnames}\NormalTok{(grid) <-}\StringTok{ }\KeywordTok{colnames}\NormalTok{(trainDf[, }\KeywordTok{c}\NormalTok{(indexA, indexB)])}
\NormalTok{  predicted.classes <-}\StringTok{ }\KeywordTok{knn}\NormalTok{(trainDf[, }\KeywordTok{c}\NormalTok{(indexA, indexB)], grid, train}\OperatorTok{$}\NormalTok{diabetes, }\DataTypeTok{k=}\NormalTok{k)}
  \KeywordTok{plot}\NormalTok{(data[, indexA], data[ ,indexB], }\DataTypeTok{pch=}\DecValTok{20}\NormalTok{, }\DataTypeTok{col=}\NormalTok{col1[}\KeywordTok{as.numeric}\NormalTok{(data}\OperatorTok{$}\NormalTok{diabetes)], }\DataTypeTok{xlab=}\KeywordTok{colnames}\NormalTok{(data)[indexA], }\DataTypeTok{ylab=}\KeywordTok{colnames}\NormalTok{(data)[indexB])}
  \KeywordTok{points}\NormalTok{(grid[, }\DecValTok{1}\NormalTok{], grid[,}\DecValTok{2}\NormalTok{], }\DataTypeTok{pch=}\StringTok{'.'}\NormalTok{, }\DataTypeTok{col=}\NormalTok{col1[}\KeywordTok{as.numeric}\NormalTok{(predicted.classes)])  }\CommentTok{# draw grid}
  \KeywordTok{legend}\NormalTok{(}\StringTok{"topleft"}\NormalTok{, }\DataTypeTok{legend=}\KeywordTok{levels}\NormalTok{(data}\OperatorTok{$}\NormalTok{diabetes),}\DataTypeTok{fill =}\NormalTok{col1)}
  \KeywordTok{title}\NormalTok{(}\KeywordTok{c}\NormalTok{(}\StringTok{"Classification with k="}\NormalTok{, k))}
\NormalTok{  predicted.matrix <-}\StringTok{ }\KeywordTok{matrix}\NormalTok{(}\KeywordTok{as.numeric}\NormalTok{(predicted.classes), }\KeywordTok{length}\NormalTok{(grid.A), }\KeywordTok{length}\NormalTok{(grid.B))}
  \KeywordTok{contour}\NormalTok{(grid.A, grid.B, predicted.matrix, }\DataTypeTok{levels=}\KeywordTok{c}\NormalTok{(}\FloatTok{1.5}\NormalTok{), }\DataTypeTok{drawlabels=}\OtherTok{FALSE}\NormalTok{,}\DataTypeTok{add=}\OtherTok{TRUE}\NormalTok{)}
\NormalTok{\}}
\end{Highlighting}
\end{Shaded}

\hypertarget{section-1}{%
\subsubsection{}\label{section-1}}

\hypertarget{glucose-and-age}{%
\paragraph{Glucose and age}\label{glucose-and-age}}

Plotting of the boundaries between columns glucose and age

\begin{Shaded}
\begin{Highlighting}[]
\KeywordTok{plot.knn}\NormalTok{(}\DecValTok{15}\NormalTok{, }\DecValTok{2}\NormalTok{, }\DecValTok{8}\NormalTok{)}
\end{Highlighting}
\end{Shaded}

\includegraphics{assigment1_files/figure-latex/unnamed-chunk-12-1.pdf}
\#\#\#\# Pressure and mass Plotting of the boundaries between columns
pressure and mass

\begin{Shaded}
\begin{Highlighting}[]
\KeywordTok{plot.knn}\NormalTok{(}\DecValTok{23}\NormalTok{, }\DecValTok{3}\NormalTok{, }\DecValTok{6}\NormalTok{)}
\end{Highlighting}
\end{Shaded}

\includegraphics{assigment1_files/figure-latex/unnamed-chunk-13-1.pdf}
\#\#\#\# Triceps and insulin Plotting of the boundaries between columns
triceps and insulin

\begin{Shaded}
\begin{Highlighting}[]
\KeywordTok{plot.knn}\NormalTok{(}\DecValTok{23}\NormalTok{, }\DecValTok{4}\NormalTok{, }\DecValTok{5}\NormalTok{)}
\end{Highlighting}
\end{Shaded}

\includegraphics{assigment1_files/figure-latex/unnamed-chunk-14-1.pdf}
\#\#\#\# Age and pedigree Plotting of the boundaries between columns age
and pedigree

\begin{Shaded}
\begin{Highlighting}[]
\KeywordTok{plot.knn}\NormalTok{(}\DecValTok{15}\NormalTok{, }\DecValTok{8}\NormalTok{, }\DecValTok{7}\NormalTok{)}
\end{Highlighting}
\end{Shaded}

\includegraphics{assigment1_files/figure-latex/unnamed-chunk-15-1.pdf}
\#\#\#Plotting the boundaties in the first two PCA components

\begin{Shaded}
\begin{Highlighting}[]
\NormalTok{res.pca <-}\StringTok{ }\KeywordTok{prcomp}\NormalTok{(data[,}\DecValTok{1}\OperatorTok{:}\DecValTok{8}\NormalTok{], }\DataTypeTok{scale =} \OtherTok{TRUE}\NormalTok{)}
\NormalTok{plot.knn.pca <-}\StringTok{ }\ControlFlowTok{function}\NormalTok{(k) \{}
\NormalTok{  pca.values <-}\StringTok{ }\KeywordTok{prcomp}\NormalTok{(trainDf)}\OperatorTok{$}\NormalTok{x}
\NormalTok{  grid.PC1 <-}\StringTok{ }\KeywordTok{seq}\NormalTok{(}\KeywordTok{min}\NormalTok{(pca.values[,}\DecValTok{1}\NormalTok{]), }\KeywordTok{max}\NormalTok{(pca.values[,}\DecValTok{1}\NormalTok{]), }\DecValTok{5}\NormalTok{)}
\NormalTok{  grid.PC2 <-}\StringTok{ }\KeywordTok{seq}\NormalTok{(}\KeywordTok{min}\NormalTok{(pca.values[,}\DecValTok{2}\NormalTok{]), }\KeywordTok{max}\NormalTok{(pca.values[,}\DecValTok{2}\NormalTok{]), }\DecValTok{1}\NormalTok{)}
\NormalTok{  grid <-}\StringTok{ }\KeywordTok{expand.grid}\NormalTok{(grid.PC1,grid.PC2)}
  \KeywordTok{colnames}\NormalTok{(grid) <-}\StringTok{ }\KeywordTok{c}\NormalTok{(}\StringTok{'PC1'}\NormalTok{,}\StringTok{'PC2'}\NormalTok{)}
\NormalTok{  predicted.classes <-}\StringTok{ }\KeywordTok{knn}\NormalTok{(pca.values[,}\KeywordTok{c}\NormalTok{(}\DecValTok{1}\NormalTok{,}\DecValTok{2}\NormalTok{)], grid, train}\OperatorTok{$}\NormalTok{diabetes, }\DataTypeTok{k=}\NormalTok{k)}
  \KeywordTok{plot}\NormalTok{(pca.values[,}\DecValTok{1}\NormalTok{], pca.values[,}\DecValTok{2}\NormalTok{], }\DataTypeTok{pch=}\DecValTok{20}\NormalTok{, }\DataTypeTok{col=}\NormalTok{col1[}\KeywordTok{as.numeric}\NormalTok{(train}\OperatorTok{$}\NormalTok{diabetes)], }\DataTypeTok{xlab=}\StringTok{'PC1'}\NormalTok{, }\DataTypeTok{ylab=}\StringTok{'PC2'}\NormalTok{)}
  \KeywordTok{points}\NormalTok{(grid}\OperatorTok{$}\NormalTok{PC1, grid}\OperatorTok{$}\NormalTok{PC2, }\DataTypeTok{pch=}\StringTok{'.'}\NormalTok{, }\DataTypeTok{col=}\NormalTok{col1[}\KeywordTok{as.numeric}\NormalTok{(predicted.classes)])  }\CommentTok{# draw grid}
  \KeywordTok{legend}\NormalTok{(}\StringTok{"topleft"}\NormalTok{, }\DataTypeTok{legend=}\KeywordTok{levels}\NormalTok{(data}\OperatorTok{$}\NormalTok{diabetes),}\DataTypeTok{fill =}\NormalTok{col1)}
  \KeywordTok{title}\NormalTok{(}\KeywordTok{c}\NormalTok{(}\StringTok{"PCA Classification with k="}\NormalTok{, k))}
  
\NormalTok{  predicted.matrix <-}\StringTok{ }\KeywordTok{matrix}\NormalTok{(}\KeywordTok{as.numeric}\NormalTok{(predicted.classes), }\KeywordTok{length}\NormalTok{(grid.PC1), }\KeywordTok{length}\NormalTok{(grid.PC2))}
  \KeywordTok{contour}\NormalTok{(grid.PC1, grid.PC2, predicted.matrix, }\DataTypeTok{levels=}\KeywordTok{c}\NormalTok{(}\FloatTok{1.5}\NormalTok{), }\DataTypeTok{drawlabels=}\OtherTok{FALSE}\NormalTok{,}\DataTypeTok{add=}\OtherTok{TRUE}\NormalTok{)}
\NormalTok{\}}
\end{Highlighting}
\end{Shaded}

\hypertarget{section-2}{%
\subsubsection{}\label{section-2}}

We tried plotting the first two principal components with a different
value of the k, and we can see that this parameter affects the
smoothness of the boundaries contour. With the higher value of k, the
contour is smoother, and the boundaries are less specific. For example,
we see that the model with k equals one is obviously overfitting the
train data.

\hypertarget{k-1}{%
\paragraph{k = 1}\label{k-1}}

\begin{Shaded}
\begin{Highlighting}[]
\KeywordTok{plot.knn.pca}\NormalTok{(}\DecValTok{1}\NormalTok{)}
\end{Highlighting}
\end{Shaded}

\includegraphics{assigment1_files/figure-latex/unnamed-chunk-17-1.pdf}

\hypertarget{k-7}{%
\paragraph{k = 7}\label{k-7}}

\begin{Shaded}
\begin{Highlighting}[]
\KeywordTok{plot.knn.pca}\NormalTok{(}\DecValTok{7}\NormalTok{)}
\end{Highlighting}
\end{Shaded}

\includegraphics{assigment1_files/figure-latex/unnamed-chunk-18-1.pdf}

\hypertarget{k-15}{%
\paragraph{k = 15}\label{k-15}}

\begin{Shaded}
\begin{Highlighting}[]
\KeywordTok{plot.knn.pca}\NormalTok{(}\DecValTok{15}\NormalTok{)}
\end{Highlighting}
\end{Shaded}

\includegraphics{assigment1_files/figure-latex/unnamed-chunk-19-1.pdf}

\hypertarget{k-23}{%
\paragraph{k = 23}\label{k-23}}

\begin{Shaded}
\begin{Highlighting}[]
\KeywordTok{plot.knn.pca}\NormalTok{(}\DecValTok{23}\NormalTok{)}
\end{Highlighting}
\end{Shaded}

\includegraphics{assigment1_files/figure-latex/unnamed-chunk-20-1.pdf}

\hypertarget{k-50}{%
\paragraph{k = 50}\label{k-50}}

\begin{Shaded}
\begin{Highlighting}[]
\KeywordTok{plot.knn.pca}\NormalTok{(}\DecValTok{50}\NormalTok{)}
\end{Highlighting}
\end{Shaded}

\includegraphics{assigment1_files/figure-latex/unnamed-chunk-21-1.pdf}

\hypertarget{k-100}{%
\paragraph{k = 100}\label{k-100}}

\begin{Shaded}
\begin{Highlighting}[]
\KeywordTok{plot.knn.pca}\NormalTok{(}\DecValTok{100}\NormalTok{)}
\end{Highlighting}
\end{Shaded}

\includegraphics{assigment1_files/figure-latex/unnamed-chunk-22-1.pdf}

\#\#Logistic regression

Firstly in the logistic regression method, we tried the model with all
columns of the data. Then we saw that the columns insulin, age,
pressure, and triceps have higher p-value then 5\%, so in the final
model, they are not significant.

\begin{Shaded}
\begin{Highlighting}[]
\NormalTok{glm <-}\StringTok{ }\KeywordTok{glm}\NormalTok{(diabetes}\OperatorTok{~}\NormalTok{.,}\DataTypeTok{family=}\KeywordTok{binomial}\NormalTok{(logit),}\DataTypeTok{data=}\NormalTok{train)}
\KeywordTok{summary}\NormalTok{(glm)}
\end{Highlighting}
\end{Shaded}

\begin{verbatim}
## 
## Call:
## glm(formula = diabetes ~ ., family = binomial(logit), data = train)
## 
## Deviance Residuals: 
##     Min       1Q   Median       3Q      Max  
## -2.7802  -0.7521  -0.4165   0.7210   2.3011  
## 
## Coefficients:
##               Estimate Std. Error z value Pr(>|z|)    
## (Intercept) -8.8065214  0.9218204  -9.553  < 2e-16 ***
## pregnant     0.0974263  0.0362116   2.690  0.00714 ** 
## glucose      0.0367428  0.0046411   7.917 2.44e-15 ***
## pressure    -0.0123576  0.0097888  -1.262  0.20680    
## triceps      0.0107334  0.0137321   0.782  0.43443    
## insulin     -0.0005421  0.0011242  -0.482  0.62966    
## mass         0.0874873  0.0221514   3.950 7.83e-05 ***
## pedigree     0.9076200  0.3398162   2.671  0.00756 ** 
## age          0.0176546  0.0110235   1.602  0.10926    
## ---
## Signif. codes:  0 '***' 0.001 '**' 0.01 '*' 0.05 '.' 0.1 ' ' 1
## 
## (Dispersion parameter for binomial family taken to be 1)
## 
##     Null deviance: 776.76  on 596  degrees of freedom
## Residual deviance: 563.64  on 588  degrees of freedom
## AIC: 581.64
## 
## Number of Fisher Scoring iterations: 5
\end{verbatim}

So we made the second model only with remaining significant variables.
Furthermore, we discovered that all the remaining variables are
significant.

\begin{Shaded}
\begin{Highlighting}[]
\NormalTok{glm2 <-}\StringTok{ }\KeywordTok{glm}\NormalTok{(diabetes}\OperatorTok{~}\NormalTok{pregnant }\OperatorTok{+}\StringTok{ }\NormalTok{glucose }\OperatorTok{+}\StringTok{ }\NormalTok{mass }\OperatorTok{+}\StringTok{ }\NormalTok{pedigree,}\DataTypeTok{family=}\KeywordTok{binomial}\NormalTok{(logit),}\DataTypeTok{data=}\NormalTok{train)}
\KeywordTok{summary}\NormalTok{(glm2)}
\end{Highlighting}
\end{Shaded}

\begin{verbatim}
## 
## Call:
## glm(formula = diabetes ~ pregnant + glucose + mass + pedigree, 
##     family = binomial(logit), data = train)
## 
## Deviance Residuals: 
##     Min       1Q   Median       3Q      Max  
## -2.8519  -0.7376  -0.4274   0.7189   2.3821  
## 
## Coefficients:
##              Estimate Std. Error z value Pr(>|z|)    
## (Intercept) -8.948893   0.779941 -11.474  < 2e-16 ***
## pregnant     0.124819   0.030341   4.114 3.89e-05 ***
## glucose      0.036290   0.003943   9.203  < 2e-16 ***
## mass         0.088487   0.016632   5.320 1.04e-07 ***
## pedigree     0.902777   0.338891   2.664  0.00772 ** 
## ---
## Signif. codes:  0 '***' 0.001 '**' 0.01 '*' 0.05 '.' 0.1 ' ' 1
## 
## (Dispersion parameter for binomial family taken to be 1)
## 
##     Null deviance: 776.76  on 596  degrees of freedom
## Residual deviance: 567.85  on 592  degrees of freedom
## AIC: 577.85
## 
## Number of Fisher Scoring iterations: 5
\end{verbatim}

\#\#\#The evaluation for the logistic regression

\begin{Shaded}
\begin{Highlighting}[]
\NormalTok{    predicted.classes.train <-}\StringTok{ }\KeywordTok{predict}\NormalTok{(glm2, trainDf, }\DataTypeTok{type =} \StringTok{"response"}\NormalTok{)}
\NormalTok{    predicted.classes.train <-}\StringTok{ }\KeywordTok{sapply}\NormalTok{(predicted.classes.train, }\ControlFlowTok{function}\NormalTok{(x) }\KeywordTok{round}\NormalTok{(x) }\OperatorTok{+}\StringTok{ }\DecValTok{1}\NormalTok{)}
\NormalTok{    confussionMatrixTrain =}\StringTok{ }\KeywordTok{table}\NormalTok{(predicted.classes.train, train}\OperatorTok{$}\NormalTok{diabetes)}
\NormalTok{    logr.accuracy.train <-}\StringTok{ }\NormalTok{(confussionMatrixTrain[}\DecValTok{1}\NormalTok{,}\DecValTok{1}\NormalTok{] }\OperatorTok{+}\StringTok{ }\NormalTok{confussionMatrixTrain[}\DecValTok{2}\NormalTok{,}\DecValTok{2}\NormalTok{]) }\OperatorTok{/}\StringTok{ }\KeywordTok{length}\NormalTok{(predicted.classes.train)}
\NormalTok{    logr.precision.train <-}\StringTok{ }\NormalTok{confussionMatrixTrain[}\DecValTok{1}\NormalTok{,}\DecValTok{1}\NormalTok{]}\OperatorTok{/}\KeywordTok{sum}\NormalTok{(confussionMatrixTrain[,}\DecValTok{1}\NormalTok{])}
\NormalTok{    logr.recall.train <-}\StringTok{ }\NormalTok{confussionMatrixTrain[}\DecValTok{1}\NormalTok{,}\DecValTok{1}\NormalTok{]}\OperatorTok{/}\KeywordTok{sum}\NormalTok{(confussionMatrixTrain[}\DecValTok{1}\NormalTok{,])}
    
\NormalTok{    predicted.classes.test <-}\StringTok{ }\KeywordTok{predict}\NormalTok{(glm2, testDf, }\DataTypeTok{type =} \StringTok{"response"}\NormalTok{)}
\NormalTok{    predicted.classes.test <-}\StringTok{ }\KeywordTok{sapply}\NormalTok{(predicted.classes.test, }\ControlFlowTok{function}\NormalTok{(x) }\KeywordTok{round}\NormalTok{(x) }\OperatorTok{+}\StringTok{ }\DecValTok{1}\NormalTok{)}
    
\NormalTok{    confussionMatrixTest <-}\StringTok{ }\KeywordTok{table}\NormalTok{(predicted.classes.test, test}\OperatorTok{$}\NormalTok{diabetes)}
\NormalTok{    logr.accuracy.test <-}\StringTok{ }\NormalTok{(confussionMatrixTest[}\DecValTok{1}\NormalTok{,}\DecValTok{1}\NormalTok{] }\OperatorTok{+}\StringTok{ }\NormalTok{confussionMatrixTest[}\DecValTok{2}\NormalTok{,}\DecValTok{2}\NormalTok{]) }\OperatorTok{/}\StringTok{ }\KeywordTok{length}\NormalTok{(predicted.classes.test)}
\NormalTok{    logr.precision.test <-}\StringTok{ }\NormalTok{confussionMatrixTest[}\DecValTok{1}\NormalTok{,}\DecValTok{1}\NormalTok{]}\OperatorTok{/}\KeywordTok{sum}\NormalTok{(confussionMatrixTest[,}\DecValTok{1}\NormalTok{])}
\NormalTok{    logr.recall.test <-}\StringTok{ }\NormalTok{confussionMatrixTest[}\DecValTok{1}\NormalTok{,}\DecValTok{1}\NormalTok{]}\OperatorTok{/}\KeywordTok{sum}\NormalTok{(confussionMatrixTest[}\DecValTok{1}\NormalTok{,])}
\end{Highlighting}
\end{Shaded}

\#\#\#\#\#Accuracy of the model is on the train dataset is 0.7788945.

\#\#\#\#\#Precision of the model is on the train dataset is 0.8935065.

\#\#\#\#\#Recall of the model is on the train dataset is 0.7908046.

\#\#\#\#\#Accuracy of the model is on the test dataset is 0.7660819.

\#\#\#\#\#Precision of the model is on the test dataset is 0.8086957.

\#\#\#\#\#Recall of the model is on the test dataset is 0.8378378.

\#\#\#Boundaries of logistic regression in different variables

\begin{Shaded}
\begin{Highlighting}[]
\NormalTok{plot.logRegression.boundaries <-}\StringTok{ }\ControlFlowTok{function}\NormalTok{(indexA, indexB) \{}
\NormalTok{  grid.A <-}\StringTok{ }\KeywordTok{seq}\NormalTok{(}\KeywordTok{min}\NormalTok{(data[,indexA]), }\KeywordTok{max}\NormalTok{(data[,indexA]), (}\KeywordTok{max}\NormalTok{(data[,indexA]) }\OperatorTok{-}\StringTok{ }\KeywordTok{min}\NormalTok{(data[,indexA])) }\OperatorTok{/}\StringTok{ }\DecValTok{100}\NormalTok{)}
\NormalTok{  grid.B <-}\StringTok{ }\KeywordTok{seq}\NormalTok{(}\KeywordTok{min}\NormalTok{(data[,indexB]), }\KeywordTok{max}\NormalTok{(data[,indexB]), (}\KeywordTok{max}\NormalTok{(data[,indexB]) }\OperatorTok{-}\StringTok{ }\KeywordTok{min}\NormalTok{(data[,indexB])) }\OperatorTok{/}\StringTok{ }\DecValTok{100}\NormalTok{)}
\NormalTok{  grid <-}\StringTok{ }\KeywordTok{expand.grid}\NormalTok{(grid.A,grid.B)}
  \KeywordTok{colnames}\NormalTok{(grid) <-}\StringTok{ }\KeywordTok{colnames}\NormalTok{(trainDf[, }\KeywordTok{c}\NormalTok{(indexA, indexB)])}
  
\NormalTok{  glm2 <-}\StringTok{ }\KeywordTok{glm}\NormalTok{(diabetes}\OperatorTok{~}\NormalTok{.,}\DataTypeTok{family=}\KeywordTok{binomial}\NormalTok{(logit),}\DataTypeTok{data =}\NormalTok{ train[, }\KeywordTok{c}\NormalTok{(indexA, indexB, }\DecValTok{9}\NormalTok{)])}
\NormalTok{  predicted.classes <-}\StringTok{ }\KeywordTok{predict}\NormalTok{(glm2, grid, }\DataTypeTok{type =} \StringTok{"response"}\NormalTok{)}
\NormalTok{  predicted.classes <-}\StringTok{ }\KeywordTok{lapply}\NormalTok{(predicted.classes, }\ControlFlowTok{function}\NormalTok{(x) }\KeywordTok{round}\NormalTok{(x) }\OperatorTok{+}\StringTok{ }\DecValTok{1}\NormalTok{)}
  
  \KeywordTok{plot}\NormalTok{(data[, indexA], data[ ,indexB], }\DataTypeTok{pch=}\DecValTok{20}\NormalTok{, }\DataTypeTok{col=}\NormalTok{col1[}\KeywordTok{as.numeric}\NormalTok{(data}\OperatorTok{$}\NormalTok{diabetes)], }\DataTypeTok{xlab=}\KeywordTok{colnames}\NormalTok{(data)[indexA], }\DataTypeTok{ylab=}\KeywordTok{colnames}\NormalTok{(data)[indexB])}
  
  \KeywordTok{points}\NormalTok{(grid[, }\DecValTok{1}\NormalTok{], grid[,}\DecValTok{2}\NormalTok{], }\DataTypeTok{pch=}\StringTok{'.'}\NormalTok{, }\DataTypeTok{col=}\NormalTok{col1[}\KeywordTok{as.numeric}\NormalTok{(predicted.classes)])  }\CommentTok{# draw grid}
  \KeywordTok{legend}\NormalTok{(}\StringTok{"topleft"}\NormalTok{, }\DataTypeTok{legend=}\KeywordTok{levels}\NormalTok{(data}\OperatorTok{$}\NormalTok{diabetes),}\DataTypeTok{fill =}\NormalTok{col1)}
  \KeywordTok{title}\NormalTok{(}\KeywordTok{c}\NormalTok{(}\StringTok{"Classification with logistic regression"}\NormalTok{))}
\NormalTok{  predicted.matrix <-}\StringTok{ }\KeywordTok{matrix}\NormalTok{(}\KeywordTok{as.numeric}\NormalTok{(predicted.classes), }\KeywordTok{length}\NormalTok{(grid.A), }\KeywordTok{length}\NormalTok{(grid.B))}
  \KeywordTok{contour}\NormalTok{(grid.A, grid.B, predicted.matrix, }\DataTypeTok{levels=}\KeywordTok{c}\NormalTok{(}\FloatTok{1.5}\NormalTok{), }\DataTypeTok{drawlabels=}\OtherTok{FALSE}\NormalTok{,}\DataTypeTok{add=}\OtherTok{TRUE}\NormalTok{)}
\NormalTok{\}}
\end{Highlighting}
\end{Shaded}

\hypertarget{section-3}{%
\subsubsection{}\label{section-3}}

\hypertarget{glucose-and-age-1}{%
\paragraph{Glucose and age}\label{glucose-and-age-1}}

Plotting of the boundaries between columns glucose and age

\begin{Shaded}
\begin{Highlighting}[]
\KeywordTok{plot.logRegression.boundaries}\NormalTok{(}\DecValTok{2}\NormalTok{, }\DecValTok{8}\NormalTok{)}
\end{Highlighting}
\end{Shaded}

\includegraphics{assigment1_files/figure-latex/unnamed-chunk-27-1.pdf}
\#\#\#\# Pressure and mass Plotting of the boundaries between columns
pressure and mass

\begin{Shaded}
\begin{Highlighting}[]
\KeywordTok{plot.logRegression.boundaries}\NormalTok{(}\DecValTok{3}\NormalTok{, }\DecValTok{6}\NormalTok{)}
\end{Highlighting}
\end{Shaded}

\includegraphics{assigment1_files/figure-latex/unnamed-chunk-28-1.pdf}
\#\#\#\# Triceps and insulin Plotting of the boundaries between columns
triceps and insulin

\begin{Shaded}
\begin{Highlighting}[]
\KeywordTok{plot.logRegression.boundaries}\NormalTok{(}\DecValTok{4}\NormalTok{, }\DecValTok{5}\NormalTok{)}
\end{Highlighting}
\end{Shaded}

\includegraphics{assigment1_files/figure-latex/unnamed-chunk-29-1.pdf}
\#\#\#\# Age and pedigree Plotting of the boundaries between columns age
and pedigree

\begin{Shaded}
\begin{Highlighting}[]
\KeywordTok{plot.logRegression.boundaries}\NormalTok{(}\DecValTok{8}\NormalTok{, }\DecValTok{7}\NormalTok{)}
\end{Highlighting}
\end{Shaded}

\includegraphics{assigment1_files/figure-latex/unnamed-chunk-30-1.pdf}
\#\#\#Bundaries of logistic regression in PCA

\begin{Shaded}
\begin{Highlighting}[]
\NormalTok{plot.logRegression.pca <-}\StringTok{ }\ControlFlowTok{function}\NormalTok{() \{}
\NormalTok{  pca.values <-}\StringTok{ }\KeywordTok{prcomp}\NormalTok{(trainDf)}\OperatorTok{$}\NormalTok{x}
\NormalTok{  pca.df <-}\StringTok{ }\KeywordTok{as.data.frame.matrix}\NormalTok{(pca.values[,}\KeywordTok{c}\NormalTok{(}\DecValTok{1}\NormalTok{,}\DecValTok{2}\NormalTok{)])}
\NormalTok{  pca.df}\OperatorTok{$}\NormalTok{diabetes <-}\StringTok{ }\NormalTok{train}\OperatorTok{$}\NormalTok{diabetes}
  
\NormalTok{  grid.PC1 <-}\StringTok{ }\KeywordTok{seq}\NormalTok{(}\KeywordTok{min}\NormalTok{(pca.values[,}\DecValTok{1}\NormalTok{]), }\KeywordTok{max}\NormalTok{(pca.values[,}\DecValTok{1}\NormalTok{]), }\DecValTok{5}\NormalTok{)}
\NormalTok{  grid.PC2 <-}\StringTok{ }\KeywordTok{seq}\NormalTok{(}\KeywordTok{min}\NormalTok{(pca.values[,}\DecValTok{2}\NormalTok{]), }\KeywordTok{max}\NormalTok{(pca.values[,}\DecValTok{2}\NormalTok{]), }\DecValTok{1}\NormalTok{)}
\NormalTok{  grid <-}\StringTok{ }\KeywordTok{expand.grid}\NormalTok{(grid.PC1,grid.PC2)}
  \KeywordTok{colnames}\NormalTok{(grid) <-}\StringTok{ }\KeywordTok{c}\NormalTok{(}\StringTok{'PC1'}\NormalTok{,}\StringTok{'PC2'}\NormalTok{)}
    
\NormalTok{  glm2 <-}\StringTok{ }\KeywordTok{glm}\NormalTok{(diabetes}\OperatorTok{~}\NormalTok{.,}\DataTypeTok{family=}\KeywordTok{binomial}\NormalTok{(logit),}\DataTypeTok{data=}\NormalTok{pca.df)}
\NormalTok{  predicted.classes <-}\StringTok{ }\KeywordTok{predict}\NormalTok{(glm2, grid, }\DataTypeTok{type =} \StringTok{"response"}\NormalTok{)}
\NormalTok{  predicted.classes <-}\StringTok{ }\KeywordTok{lapply}\NormalTok{(predicted.classes, }\ControlFlowTok{function}\NormalTok{(x) }\KeywordTok{round}\NormalTok{(x) }\OperatorTok{+}\StringTok{ }\DecValTok{1}\NormalTok{)}
  \KeywordTok{plot}\NormalTok{(pca.values[,}\DecValTok{1}\NormalTok{], pca.values[,}\DecValTok{2}\NormalTok{], }\DataTypeTok{pch=}\DecValTok{20}\NormalTok{, }\DataTypeTok{col=}\NormalTok{col1[}\KeywordTok{as.numeric}\NormalTok{(train}\OperatorTok{$}\NormalTok{diabetes)], }\DataTypeTok{xlab=}\StringTok{'PC1'}\NormalTok{, }\DataTypeTok{ylab=}\StringTok{'PC2'}\NormalTok{)}
  \KeywordTok{points}\NormalTok{(grid}\OperatorTok{$}\NormalTok{PC1, grid}\OperatorTok{$}\NormalTok{PC2, }\DataTypeTok{pch=}\StringTok{'.'}\NormalTok{, }\DataTypeTok{col=}\NormalTok{col1[}\KeywordTok{as.numeric}\NormalTok{(predicted.classes)])  }\CommentTok{# draw grid}
  \KeywordTok{legend}\NormalTok{(}\StringTok{"topleft"}\NormalTok{, }\DataTypeTok{legend=}\KeywordTok{levels}\NormalTok{(data}\OperatorTok{$}\NormalTok{diabetes),}\DataTypeTok{fill =}\NormalTok{col1)}
  \KeywordTok{title}\NormalTok{(}\KeywordTok{c}\NormalTok{(}\StringTok{"PCA Classification of loggistic regression"}\NormalTok{))}
  
\NormalTok{  predicted.matrix <-}\StringTok{ }\KeywordTok{matrix}\NormalTok{(}\KeywordTok{as.numeric}\NormalTok{(predicted.classes), }\KeywordTok{length}\NormalTok{(grid.PC1), }\KeywordTok{length}\NormalTok{(grid.PC2))}
  \KeywordTok{contour}\NormalTok{(grid.PC1, grid.PC2, predicted.matrix, }\DataTypeTok{levels=}\KeywordTok{c}\NormalTok{(}\FloatTok{1.5}\NormalTok{), }\DataTypeTok{drawlabels=}\OtherTok{FALSE}\NormalTok{,}\DataTypeTok{add=}\OtherTok{TRUE}\NormalTok{)}
\NormalTok{\}}
\KeywordTok{plot.logRegression.pca}\NormalTok{()}
\end{Highlighting}
\end{Shaded}

\includegraphics{assigment1_files/figure-latex/unnamed-chunk-31-1.pdf}

\#\#CART

In the cart method, we used for restraining the depth of the tree the
parameter minbucket, which controls the minimum size of the leaves of
the tree. Then we plot the test and train error for each bucket to find
out the best value for the parameter minbucket.

\begin{Shaded}
\begin{Highlighting}[]
\NormalTok{possible_bucket_sizes <-}\StringTok{ }\KeywordTok{seq}\NormalTok{(}\DecValTok{1}\NormalTok{, }\DecValTok{300}\NormalTok{, }\DecValTok{10}\NormalTok{)}
\NormalTok{nks <-}\StringTok{ }\KeywordTok{length}\NormalTok{(possible_bucket_sizes)}
\NormalTok{accuracy.train <-}\StringTok{ }\KeywordTok{numeric}\NormalTok{(}\DataTypeTok{length=}\NormalTok{nks)}
\NormalTok{accuracy.test <-}\StringTok{ }\KeywordTok{numeric}\NormalTok{(}\DataTypeTok{length=}\NormalTok{nks)}
\NormalTok{precision.train <-}\StringTok{ }\KeywordTok{numeric}\NormalTok{(}\DataTypeTok{length=}\NormalTok{nks)}
\NormalTok{precision.test <-}\StringTok{ }\KeywordTok{numeric}\NormalTok{(}\DataTypeTok{length=}\NormalTok{nks)}
\NormalTok{recall.train <-}\StringTok{ }\KeywordTok{numeric}\NormalTok{(}\DataTypeTok{length=}\NormalTok{nks)}
\NormalTok{recall.test <-}\StringTok{ }\KeywordTok{numeric}\NormalTok{(}\DataTypeTok{length=}\NormalTok{nks)}
\ControlFlowTok{for}\NormalTok{(i }\ControlFlowTok{in} \KeywordTok{seq}\NormalTok{(}\DecValTok{1}\OperatorTok{:}\KeywordTok{length}\NormalTok{(possible_bucket_sizes)))\{}
\NormalTok{  cart_model <-}\StringTok{ }\KeywordTok{rpart}\NormalTok{(diabetes }\OperatorTok{~}\NormalTok{., }\DataTypeTok{data =}\NormalTok{ train, }\DataTypeTok{method =} \StringTok{"class"}\NormalTok{,}\DataTypeTok{control =} \KeywordTok{rpart.control}\NormalTok{(}\DataTypeTok{minbucket =}\NormalTok{ possible_bucket_sizes[i]))}
\NormalTok{  predicted.classes.test <-}\StringTok{ }\KeywordTok{predict}\NormalTok{(cart_model, testDf, }\DataTypeTok{type =} \StringTok{"class"}\NormalTok{)}
\NormalTok{  predicted.classes.train <-}\StringTok{ }\KeywordTok{predict}\NormalTok{(cart_model, trainDf, }\DataTypeTok{type =} \StringTok{"class"}\NormalTok{)}
\NormalTok{    confussionMatrixTrain =}\StringTok{ }\KeywordTok{table}\NormalTok{(predicted.classes.train, train}\OperatorTok{$}\NormalTok{diabetes)}
\NormalTok{    accuracy.train[i] <-}\StringTok{ }\KeywordTok{mean}\NormalTok{(predicted.classes.train }\OperatorTok{==}\StringTok{ }\NormalTok{train}\OperatorTok{$}\NormalTok{diabetes)}
\NormalTok{    precision.train[i] =}\StringTok{ }\NormalTok{confussionMatrixTrain[}\DecValTok{1}\NormalTok{,}\DecValTok{1}\NormalTok{]}\OperatorTok{/}\KeywordTok{sum}\NormalTok{(confussionMatrixTrain[,}\DecValTok{1}\NormalTok{])}
\NormalTok{    recall.train[i] =}\StringTok{ }\NormalTok{confussionMatrixTrain[}\DecValTok{1}\NormalTok{,}\DecValTok{1}\NormalTok{]}\OperatorTok{/}\KeywordTok{sum}\NormalTok{(confussionMatrixTrain[}\DecValTok{1}\NormalTok{,])}
\NormalTok{    confussionMatrixTest =}\StringTok{ }\KeywordTok{table}\NormalTok{(predicted.classes.test, test}\OperatorTok{$}\NormalTok{diabetes)}
\NormalTok{    accuracy.test[i] <-}\StringTok{ }\KeywordTok{mean}\NormalTok{(predicted.classes.test }\OperatorTok{==}\StringTok{ }\NormalTok{test}\OperatorTok{$}\NormalTok{diabetes)}
\NormalTok{    precision.test[i] =}\StringTok{ }\NormalTok{confussionMatrixTest[}\DecValTok{1}\NormalTok{,}\DecValTok{1}\NormalTok{]}\OperatorTok{/}\KeywordTok{sum}\NormalTok{(confussionMatrixTest[,}\DecValTok{1}\NormalTok{])}
\NormalTok{    recall.test[i] =}\StringTok{ }\NormalTok{confussionMatrixTest[}\DecValTok{1}\NormalTok{,}\DecValTok{1}\NormalTok{]}\OperatorTok{/}\KeywordTok{sum}\NormalTok{(confussionMatrixTest[}\DecValTok{1}\NormalTok{,])}
\NormalTok{\}}
\end{Highlighting}
\end{Shaded}

\hypertarget{section-4}{%
\subsubsection{}\label{section-4}}

When we compare the train and the test line in the plot, we see that all
evaluation metrics are in the train line at the maxim with k equals 1,
because we return the real value.

\hypertarget{accuracy-1}{%
\paragraph{Accuracy}\label{accuracy-1}}

The accuracy achieves its maximum on the test dataset with k equals 15.

\begin{Shaded}
\begin{Highlighting}[]
\KeywordTok{plot}\NormalTok{(accuracy.train,}\DataTypeTok{xlab=}\StringTok{"Number of items in leafs"}\NormalTok{,}\DataTypeTok{ylab=}\StringTok{"Test error"}\NormalTok{,}\DataTypeTok{type=}\StringTok{"n"}\NormalTok{,}\DataTypeTok{xaxt=}\StringTok{"n"}\NormalTok{, }\DataTypeTok{ylim=}\KeywordTok{c}\NormalTok{(}\FloatTok{0.5}\NormalTok{, }\DecValTok{1}\NormalTok{),  }\DataTypeTok{main=}\StringTok{"Accuracy"}\NormalTok{,)}
\KeywordTok{axis}\NormalTok{(}\DecValTok{1}\NormalTok{, }\DecValTok{1}\OperatorTok{:}\KeywordTok{length}\NormalTok{(ks), }\KeywordTok{as.character}\NormalTok{(ks))}
\KeywordTok{lines}\NormalTok{(accuracy.train,}\DataTypeTok{type=}\StringTok{"b"}\NormalTok{,}\DataTypeTok{col=}\StringTok{'red'}\NormalTok{,}\DataTypeTok{pch=}\DecValTok{20}\NormalTok{)}
\KeywordTok{lines}\NormalTok{(accuracy.test,}\DataTypeTok{type=}\StringTok{"b"}\NormalTok{,}\DataTypeTok{col=}\StringTok{'blue'}\NormalTok{,}\DataTypeTok{pch=}\DecValTok{20}\NormalTok{)}
\KeywordTok{legend}\NormalTok{(}\StringTok{"bottomright"}\NormalTok{,}\DataTypeTok{lty=}\DecValTok{1}\NormalTok{,}\DataTypeTok{col=}\KeywordTok{c}\NormalTok{(}\StringTok{"red"}\NormalTok{,}\StringTok{"blue"}\NormalTok{),}\DataTypeTok{legend =} \KeywordTok{c}\NormalTok{(}\StringTok{"train "}\NormalTok{, }\StringTok{"test "}\NormalTok{))}
\end{Highlighting}
\end{Shaded}

\includegraphics{assigment1_files/figure-latex/unnamed-chunk-33-1.pdf}

\hypertarget{precision-1}{%
\paragraph{Precision}\label{precision-1}}

The precision achieves its maximum on the test dataset with k equals 23.

\begin{Shaded}
\begin{Highlighting}[]
\KeywordTok{plot}\NormalTok{(precision.train,}\DataTypeTok{xlab=}\StringTok{"Number of items in leafs"}\NormalTok{,}\DataTypeTok{ylab=}\StringTok{"Test error"}\NormalTok{,}\DataTypeTok{type=}\StringTok{"n"}\NormalTok{,}\DataTypeTok{xaxt=}\StringTok{"n"}\NormalTok{, }\DataTypeTok{ylim=}\KeywordTok{c}\NormalTok{(}\FloatTok{0.5}\NormalTok{, }\DecValTok{1}\NormalTok{),  }\DataTypeTok{main=}\StringTok{"Precision"}\NormalTok{,)}
\KeywordTok{axis}\NormalTok{(}\DecValTok{1}\NormalTok{, }\DecValTok{1}\OperatorTok{:}\KeywordTok{length}\NormalTok{(ks), }\KeywordTok{as.character}\NormalTok{(ks))}
\KeywordTok{lines}\NormalTok{(precision.train,}\DataTypeTok{type=}\StringTok{"b"}\NormalTok{,}\DataTypeTok{col=}\StringTok{'red'}\NormalTok{,}\DataTypeTok{pch=}\DecValTok{20}\NormalTok{)}
\KeywordTok{lines}\NormalTok{(precision.test,}\DataTypeTok{type=}\StringTok{"b"}\NormalTok{,}\DataTypeTok{col=}\StringTok{'blue'}\NormalTok{,}\DataTypeTok{pch=}\DecValTok{20}\NormalTok{)}
\KeywordTok{legend}\NormalTok{(}\StringTok{"bottomright"}\NormalTok{,}\DataTypeTok{lty=}\DecValTok{1}\NormalTok{,}\DataTypeTok{col=}\KeywordTok{c}\NormalTok{(}\StringTok{"red"}\NormalTok{,}\StringTok{"blue"}\NormalTok{),}\DataTypeTok{legend =} \KeywordTok{c}\NormalTok{(}\StringTok{"train "}\NormalTok{, }\StringTok{"test "}\NormalTok{))}
\end{Highlighting}
\end{Shaded}

\includegraphics{assigment1_files/figure-latex/unnamed-chunk-34-1.pdf}

\hypertarget{recall-1}{%
\paragraph{Recall}\label{recall-1}}

The recall achieves its maximum on the test dataset with k equals 15.

\begin{Shaded}
\begin{Highlighting}[]
\KeywordTok{plot}\NormalTok{(recall.train,}\DataTypeTok{xlab=}\StringTok{"Number of items in leafs"}\NormalTok{,}\DataTypeTok{ylab=}\StringTok{"Test error"}\NormalTok{,}\DataTypeTok{type=}\StringTok{"n"}\NormalTok{,}\DataTypeTok{xaxt=}\StringTok{"n"}\NormalTok{, }\DataTypeTok{ylim=}\KeywordTok{c}\NormalTok{(}\FloatTok{0.5}\NormalTok{, }\DecValTok{1}\NormalTok{),  }\DataTypeTok{main=}\StringTok{"Recall"}\NormalTok{,)}
\KeywordTok{axis}\NormalTok{(}\DecValTok{1}\NormalTok{, }\DecValTok{1}\OperatorTok{:}\KeywordTok{length}\NormalTok{(ks), }\KeywordTok{as.character}\NormalTok{(ks))}
\KeywordTok{lines}\NormalTok{(recall.train,}\DataTypeTok{type=}\StringTok{"b"}\NormalTok{,}\DataTypeTok{col=}\StringTok{'red'}\NormalTok{,}\DataTypeTok{pch=}\DecValTok{20}\NormalTok{)}
\KeywordTok{lines}\NormalTok{(recall.test,}\DataTypeTok{type=}\StringTok{"b"}\NormalTok{,}\DataTypeTok{col=}\StringTok{'blue'}\NormalTok{,}\DataTypeTok{pch=}\DecValTok{20}\NormalTok{)}
\KeywordTok{legend}\NormalTok{(}\StringTok{"bottomright"}\NormalTok{,}\DataTypeTok{lty=}\DecValTok{1}\NormalTok{,}\DataTypeTok{col=}\KeywordTok{c}\NormalTok{(}\StringTok{"red"}\NormalTok{,}\StringTok{"blue"}\NormalTok{),}\DataTypeTok{legend =} \KeywordTok{c}\NormalTok{(}\StringTok{"train "}\NormalTok{, }\StringTok{"test "}\NormalTok{))}
\end{Highlighting}
\end{Shaded}

\includegraphics{assigment1_files/figure-latex/unnamed-chunk-35-1.pdf}

\begin{Shaded}
\begin{Highlighting}[]
\NormalTok{plot.treeBoundaries <-}\StringTok{ }\ControlFlowTok{function}\NormalTok{(mb, indexA, indexB) \{}
\NormalTok{  grid.A <-}\StringTok{ }\KeywordTok{seq}\NormalTok{(}\KeywordTok{min}\NormalTok{(data[,indexA]), }\KeywordTok{max}\NormalTok{(data[,indexA]), (}\KeywordTok{max}\NormalTok{(data[,indexA]) }\OperatorTok{-}\StringTok{ }\KeywordTok{min}\NormalTok{(data[,indexA])) }\OperatorTok{/}\StringTok{ }\DecValTok{100}\NormalTok{)}
\NormalTok{  grid.B <-}\StringTok{ }\KeywordTok{seq}\NormalTok{(}\KeywordTok{min}\NormalTok{(data[,indexB]), }\KeywordTok{max}\NormalTok{(data[,indexB]), (}\KeywordTok{max}\NormalTok{(data[,indexB]) }\OperatorTok{-}\StringTok{ }\KeywordTok{min}\NormalTok{(data[,indexB])) }\OperatorTok{/}\StringTok{ }\DecValTok{100}\NormalTok{)}
\NormalTok{  grid <-}\StringTok{ }\KeywordTok{expand.grid}\NormalTok{(grid.A,grid.B)}
  \KeywordTok{colnames}\NormalTok{(grid) <-}\StringTok{ }\KeywordTok{colnames}\NormalTok{(trainDf[, }\KeywordTok{c}\NormalTok{(indexA, indexB)])}
  
\NormalTok{  cart_model <-}\StringTok{ }\KeywordTok{rpart}\NormalTok{(diabetes }\OperatorTok{~}\StringTok{ }\NormalTok{. , }\DataTypeTok{data =}\NormalTok{ train[, }\KeywordTok{c}\NormalTok{(indexA, indexB, }\DecValTok{9}\NormalTok{)], }\DataTypeTok{method =} \StringTok{"class"}\NormalTok{,}\DataTypeTok{control =} \KeywordTok{rpart.control}\NormalTok{(}\DataTypeTok{minbucket =}\NormalTok{ mb))}
\NormalTok{  predicted.classes <-}\StringTok{ }\KeywordTok{predict}\NormalTok{(cart_model, grid, }\DataTypeTok{type =} \StringTok{"class"}\NormalTok{)}
  
  \KeywordTok{plot}\NormalTok{(data[, indexA], data[ ,indexB], }\DataTypeTok{pch=}\DecValTok{20}\NormalTok{, }\DataTypeTok{col=}\NormalTok{col1[}\KeywordTok{as.numeric}\NormalTok{(data}\OperatorTok{$}\NormalTok{diabetes)], }\DataTypeTok{xlab=}\KeywordTok{colnames}\NormalTok{(data)[indexA], }\DataTypeTok{ylab=}\KeywordTok{colnames}\NormalTok{(data)[indexB])}
  
  \KeywordTok{points}\NormalTok{(grid[, }\DecValTok{1}\NormalTok{], grid[,}\DecValTok{2}\NormalTok{], }\DataTypeTok{pch=}\StringTok{'.'}\NormalTok{, }\DataTypeTok{col=}\NormalTok{col1[}\KeywordTok{as.numeric}\NormalTok{(predicted.classes)])  }\CommentTok{# draw grid}
  \KeywordTok{legend}\NormalTok{(}\StringTok{"topleft"}\NormalTok{, }\DataTypeTok{legend=}\KeywordTok{levels}\NormalTok{(data}\OperatorTok{$}\NormalTok{diabetes),}\DataTypeTok{fill =}\NormalTok{col1)}
  \KeywordTok{title}\NormalTok{(}\KeywordTok{c}\NormalTok{(}\StringTok{"Classification with minibucket size="}\NormalTok{, mb))}
\NormalTok{  predicted.matrix <-}\StringTok{ }\KeywordTok{matrix}\NormalTok{(}\KeywordTok{as.numeric}\NormalTok{(predicted.classes), }\KeywordTok{length}\NormalTok{(grid.A), }\KeywordTok{length}\NormalTok{(grid.B))}
  \KeywordTok{contour}\NormalTok{(grid.A, grid.B, predicted.matrix, }\DataTypeTok{levels=}\KeywordTok{c}\NormalTok{(}\FloatTok{1.5}\NormalTok{), }\DataTypeTok{drawlabels=}\OtherTok{FALSE}\NormalTok{,}\DataTypeTok{add=}\OtherTok{TRUE}\NormalTok{)}
\NormalTok{\}}
\end{Highlighting}
\end{Shaded}

\hypertarget{section-5}{%
\subsubsection{}\label{section-5}}

\hypertarget{glucose-and-age-2}{%
\paragraph{Glucose and age}\label{glucose-and-age-2}}

Plotting of the boundaries between columns glucose and age

\begin{Shaded}
\begin{Highlighting}[]
\KeywordTok{plot.treeBoundaries}\NormalTok{(}\DecValTok{23}\NormalTok{, }\DecValTok{2}\NormalTok{, }\DecValTok{8}\NormalTok{)}
\end{Highlighting}
\end{Shaded}

\includegraphics{assigment1_files/figure-latex/unnamed-chunk-37-1.pdf}
\#\#\#\# Pressure and mass Plotting of the boundaries between columns
pressure and mass

\begin{Shaded}
\begin{Highlighting}[]
\KeywordTok{plot.treeBoundaries}\NormalTok{(}\DecValTok{23}\NormalTok{,}\DecValTok{3}\NormalTok{, }\DecValTok{6}\NormalTok{)}
\end{Highlighting}
\end{Shaded}

\includegraphics{assigment1_files/figure-latex/unnamed-chunk-38-1.pdf}
\#\#\#\# Triceps and insulin Plotting of the boundaries between columns
triceps and insulin

\begin{Shaded}
\begin{Highlighting}[]
\KeywordTok{plot.treeBoundaries}\NormalTok{(}\DecValTok{23}\NormalTok{,}\DecValTok{4}\NormalTok{, }\DecValTok{5}\NormalTok{)}
\end{Highlighting}
\end{Shaded}

\includegraphics{assigment1_files/figure-latex/unnamed-chunk-39-1.pdf}
\#\#\#\# Age and pedigree Plotting of the boundaries between columns age
and pedigree

\begin{Shaded}
\begin{Highlighting}[]
\KeywordTok{plot.treeBoundaries}\NormalTok{(}\DecValTok{23}\NormalTok{, }\DecValTok{8}\NormalTok{, }\DecValTok{7}\NormalTok{)}
\end{Highlighting}
\end{Shaded}

\includegraphics{assigment1_files/figure-latex/unnamed-chunk-40-1.pdf}
\#\#\#Plotting first two components of PCA

\begin{Shaded}
\begin{Highlighting}[]
\NormalTok{plot.treeBoundaries.pca <-}\StringTok{ }\ControlFlowTok{function}\NormalTok{(mb) \{}
\NormalTok{  pca.values <-}\StringTok{ }\KeywordTok{prcomp}\NormalTok{(trainDf)}\OperatorTok{$}\NormalTok{x}
\NormalTok{  pca.df <-}\StringTok{ }\KeywordTok{as.data.frame.matrix}\NormalTok{(pca.values[,}\KeywordTok{c}\NormalTok{(}\DecValTok{1}\NormalTok{,}\DecValTok{2}\NormalTok{)])}
\NormalTok{  pca.df}\OperatorTok{$}\NormalTok{diabetes <-}\StringTok{ }\NormalTok{train}\OperatorTok{$}\NormalTok{diabetes}
  
\NormalTok{  grid.PC1 <-}\StringTok{ }\KeywordTok{seq}\NormalTok{(}\KeywordTok{min}\NormalTok{(pca.values[,}\DecValTok{1}\NormalTok{]), }\KeywordTok{max}\NormalTok{(pca.values[,}\DecValTok{1}\NormalTok{]), }\DecValTok{5}\NormalTok{)}
\NormalTok{  grid.PC2 <-}\StringTok{ }\KeywordTok{seq}\NormalTok{(}\KeywordTok{min}\NormalTok{(pca.values[,}\DecValTok{2}\NormalTok{]), }\KeywordTok{max}\NormalTok{(pca.values[,}\DecValTok{2}\NormalTok{]), }\DecValTok{1}\NormalTok{)}
\NormalTok{  grid <-}\StringTok{ }\KeywordTok{expand.grid}\NormalTok{(grid.PC1,grid.PC2)}
  \KeywordTok{colnames}\NormalTok{(grid) <-}\StringTok{ }\KeywordTok{c}\NormalTok{(}\StringTok{'PC1'}\NormalTok{,}\StringTok{'PC2'}\NormalTok{)}
    
\NormalTok{  cart_model <-}\StringTok{ }\KeywordTok{rpart}\NormalTok{(diabetes }\OperatorTok{~}\NormalTok{., }\DataTypeTok{data =}\NormalTok{ pca.df, }\DataTypeTok{method =} \StringTok{"class"}\NormalTok{,}\DataTypeTok{control =} \KeywordTok{rpart.control}\NormalTok{(}\DataTypeTok{minbucket =}\NormalTok{ mb))}
\NormalTok{  predicted.classes <-}\StringTok{ }\KeywordTok{predict}\NormalTok{(cart_model, grid, }\DataTypeTok{type =} \StringTok{"class"}\NormalTok{)}
  \KeywordTok{plot}\NormalTok{(pca.values[,}\DecValTok{1}\NormalTok{], pca.values[,}\DecValTok{2}\NormalTok{], }\DataTypeTok{pch=}\DecValTok{20}\NormalTok{, }\DataTypeTok{col=}\NormalTok{col1[}\KeywordTok{as.numeric}\NormalTok{(train}\OperatorTok{$}\NormalTok{diabetes)], }\DataTypeTok{xlab=}\StringTok{'PC1'}\NormalTok{, }\DataTypeTok{ylab=}\StringTok{'PC2'}\NormalTok{)}
  \KeywordTok{points}\NormalTok{(grid}\OperatorTok{$}\NormalTok{PC1, grid}\OperatorTok{$}\NormalTok{PC2, }\DataTypeTok{pch=}\StringTok{'.'}\NormalTok{, }\DataTypeTok{col=}\NormalTok{col1[}\KeywordTok{as.numeric}\NormalTok{(predicted.classes)])  }\CommentTok{# draw grid}
  \KeywordTok{legend}\NormalTok{(}\StringTok{"topleft"}\NormalTok{, }\DataTypeTok{legend=}\KeywordTok{levels}\NormalTok{(data}\OperatorTok{$}\NormalTok{diabetes),}\DataTypeTok{fill =}\NormalTok{col1)}
  \KeywordTok{title}\NormalTok{(}\KeywordTok{c}\NormalTok{(}\StringTok{"PCA Classification with mini bucket size ="}\NormalTok{, mb))}
  
\NormalTok{  predicted.matrix <-}\StringTok{ }\KeywordTok{matrix}\NormalTok{(}\KeywordTok{as.numeric}\NormalTok{(predicted.classes), }\KeywordTok{length}\NormalTok{(grid.PC1), }\KeywordTok{length}\NormalTok{(grid.PC2))}
  \KeywordTok{contour}\NormalTok{(grid.PC1, grid.PC2, predicted.matrix, }\DataTypeTok{levels=}\KeywordTok{c}\NormalTok{(}\FloatTok{1.5}\NormalTok{), }\DataTypeTok{drawlabels=}\OtherTok{FALSE}\NormalTok{,}\DataTypeTok{add=}\OtherTok{TRUE}\NormalTok{)}
\NormalTok{\}}
\end{Highlighting}
\end{Shaded}

\hypertarget{section-6}{%
\subsubsection{}\label{section-6}}

We tried plotting the first two principal components with a different
value of the minbucket. We see that with a lower value, the model is
overfitting the data. We can see it in the plots with minibucket equals
1 and 7. Furthermore, with minbucket with value more than 50, we see
that the tree consists only of the one node, and its probably
underfitting the data.

\hypertarget{k-1-1}{%
\paragraph{k = 1}\label{k-1-1}}

\begin{Shaded}
\begin{Highlighting}[]
\KeywordTok{plot.treeBoundaries.pca}\NormalTok{(}\DecValTok{1}\NormalTok{)}
\end{Highlighting}
\end{Shaded}

\includegraphics{assigment1_files/figure-latex/unnamed-chunk-42-1.pdf}

\hypertarget{k-7-1}{%
\paragraph{k = 7}\label{k-7-1}}

\begin{Shaded}
\begin{Highlighting}[]
\KeywordTok{plot.treeBoundaries.pca}\NormalTok{(}\DecValTok{7}\NormalTok{)}
\end{Highlighting}
\end{Shaded}

\includegraphics{assigment1_files/figure-latex/unnamed-chunk-43-1.pdf}

\hypertarget{k-15-1}{%
\paragraph{k = 15}\label{k-15-1}}

\begin{Shaded}
\begin{Highlighting}[]
\KeywordTok{plot.treeBoundaries.pca}\NormalTok{(}\DecValTok{15}\NormalTok{)}
\end{Highlighting}
\end{Shaded}

\includegraphics{assigment1_files/figure-latex/unnamed-chunk-44-1.pdf}

\hypertarget{k-23-1}{%
\paragraph{k = 23}\label{k-23-1}}

\begin{Shaded}
\begin{Highlighting}[]
\KeywordTok{plot.treeBoundaries.pca}\NormalTok{(}\DecValTok{23}\NormalTok{)}
\end{Highlighting}
\end{Shaded}

\includegraphics{assigment1_files/figure-latex/unnamed-chunk-45-1.pdf}

\hypertarget{k-35}{%
\paragraph{k = 35}\label{k-35}}

\begin{Shaded}
\begin{Highlighting}[]
\KeywordTok{plot.treeBoundaries.pca}\NormalTok{(}\DecValTok{35}\NormalTok{)}
\end{Highlighting}
\end{Shaded}

\includegraphics{assigment1_files/figure-latex/unnamed-chunk-46-1.pdf}

\hypertarget{k-50-1}{%
\paragraph{k = 50}\label{k-50-1}}

\begin{Shaded}
\begin{Highlighting}[]
\KeywordTok{plot.treeBoundaries.pca}\NormalTok{(}\DecValTok{50}\NormalTok{)}
\end{Highlighting}
\end{Shaded}

\includegraphics{assigment1_files/figure-latex/unnamed-chunk-47-1.pdf}
\#\# Linear regression

Then we explored the linear regression method. Firstly we constructed a
model with all the predictors and we saw that only pregnant, glucose,
mass and pedegree were significant.

\begin{Shaded}
\begin{Highlighting}[]
\NormalTok{ml<-}\KeywordTok{lm}\NormalTok{(}\KeywordTok{as.numeric}\NormalTok{(diabetes)}\OperatorTok{~}\NormalTok{pregnant}\OperatorTok{+}\NormalTok{glucose}\OperatorTok{+}\NormalTok{pressure}\OperatorTok{+}\NormalTok{triceps}\OperatorTok{+}\NormalTok{insulin}\OperatorTok{+}\NormalTok{mass}\OperatorTok{+}\NormalTok{pedigree}\OperatorTok{+}\NormalTok{age, }\DataTypeTok{data=}\NormalTok{train)}
\KeywordTok{summary}\NormalTok{(ml)}
\end{Highlighting}
\end{Shaded}

\begin{verbatim}
## 
## Call:
## lm(formula = as.numeric(diabetes) ~ pregnant + glucose + pressure + 
##     triceps + insulin + mass + pedigree + age, data = train)
## 
## Residuals:
##      Min       1Q   Median       3Q      Max 
## -1.13556 -0.29971 -0.08002  0.29988  0.95966 
## 
## Coefficients:
##               Estimate Std. Error t value Pr(>|t|)    
## (Intercept) -1.693e-02  1.221e-01  -0.139  0.88974    
## pregnant     1.691e-02  5.752e-03   2.940  0.00341 ** 
## glucose      6.486e-03  6.788e-04   9.555  < 2e-16 ***
## pressure    -1.742e-03  1.514e-03  -1.151  0.25034    
## triceps      1.510e-03  2.109e-03   0.716  0.47419    
## insulin     -8.739e-05  1.893e-04  -0.462  0.64458    
## mass         1.432e-02  3.343e-03   4.284 2.14e-05 ***
## pedigree     1.409e-01  5.278e-02   2.670  0.00779 ** 
## age          2.620e-03  1.770e-03   1.480  0.13942    
## ---
## Signif. codes:  0 '***' 0.001 '**' 0.01 '*' 0.05 '.' 0.1 ' ' 1
## 
## Residual standard error: 0.3987 on 588 degrees of freedom
## Multiple R-squared:  0.3162, Adjusted R-squared:  0.3069 
## F-statistic: 33.98 on 8 and 588 DF,  p-value: < 2.2e-16
\end{verbatim}

So we made a new model with the relevant variables. As we can see, they
remain significant, so this is our final model.

\begin{Shaded}
\begin{Highlighting}[]
\NormalTok{ml2<-}\KeywordTok{lm}\NormalTok{(}\KeywordTok{as.numeric}\NormalTok{(diabetes)}\OperatorTok{~}\NormalTok{pregnant}\OperatorTok{+}\NormalTok{glucose}\OperatorTok{+}\StringTok{ }\NormalTok{mass}\OperatorTok{+}\NormalTok{pedigree, }\DataTypeTok{data=}\NormalTok{train)}
\KeywordTok{summary}\NormalTok{(ml2)}
\end{Highlighting}
\end{Shaded}

\begin{verbatim}
## 
## Call:
## lm(formula = as.numeric(diabetes) ~ pregnant + glucose + mass + 
##     pedigree, data = train)
## 
## Residuals:
##     Min      1Q  Median      3Q     Max 
## -1.1765 -0.2902 -0.0836  0.2963  0.9928 
## 
## Coefficients:
##               Estimate Std. Error t value Pr(>|t|)    
## (Intercept) -0.0488139  0.0945913  -0.516   0.6060    
## pregnant     0.0210760  0.0048066   4.385 1.37e-05 ***
## glucose      0.0064667  0.0005599  11.551  < 2e-16 ***
## mass         0.0146206  0.0024921   5.867 7.40e-09 ***
## pedigree     0.1418975  0.0525227   2.702   0.0071 ** 
## ---
## Signif. codes:  0 '***' 0.001 '**' 0.01 '*' 0.05 '.' 0.1 ' ' 1
## 
## Residual standard error: 0.3986 on 592 degrees of freedom
## Multiple R-squared:  0.312,  Adjusted R-squared:  0.3073 
## F-statistic:  67.1 on 4 and 592 DF,  p-value: < 2.2e-16
\end{verbatim}

\#\#\#Evaluation for linear regression

To look at the expressiveness of our model we computed different
evaluation metris (accuracy, precision and recall).

\begin{Shaded}
\begin{Highlighting}[]
\NormalTok{    predicted.classes.train <-}\StringTok{ }\KeywordTok{predict}\NormalTok{(ml2, trainDf, }\DataTypeTok{type =} \StringTok{"response"}\NormalTok{)}
\NormalTok{    predicted.classes.train <-}\StringTok{ }\KeywordTok{sapply}\NormalTok{(predicted.classes.train, }\ControlFlowTok{function}\NormalTok{(x) }\KeywordTok{round}\NormalTok{(x))}
\NormalTok{    confussionMatrixTrain =}\StringTok{ }\KeywordTok{table}\NormalTok{(predicted.classes.train, train}\OperatorTok{$}\NormalTok{diabetes)}
\NormalTok{    linreg.accuracy.train <-}\StringTok{ }\NormalTok{(confussionMatrixTrain[}\DecValTok{1}\NormalTok{,}\DecValTok{1}\NormalTok{] }\OperatorTok{+}\StringTok{ }\NormalTok{confussionMatrixTrain[}\DecValTok{2}\NormalTok{,}\DecValTok{2}\NormalTok{]) }\OperatorTok{/}\StringTok{ }\KeywordTok{length}\NormalTok{(predicted.classes.train)}
\NormalTok{    linreg.precision.train <-}\StringTok{ }\NormalTok{confussionMatrixTrain[}\DecValTok{1}\NormalTok{,}\DecValTok{1}\NormalTok{]}\OperatorTok{/}\KeywordTok{sum}\NormalTok{(confussionMatrixTrain[,}\DecValTok{1}\NormalTok{])}
\NormalTok{    linreg.recall.train <-}\StringTok{ }\NormalTok{confussionMatrixTrain[}\DecValTok{1}\NormalTok{,}\DecValTok{1}\NormalTok{]}\OperatorTok{/}\KeywordTok{sum}\NormalTok{(confussionMatrixTrain[}\DecValTok{1}\NormalTok{,])}
    
\NormalTok{    predicted.classes.test <-}\StringTok{ }\KeywordTok{predict}\NormalTok{(ml2, testDf, }\DataTypeTok{type =} \StringTok{"response"}\NormalTok{)}
\NormalTok{    predicted.classes.test <-}\StringTok{ }\KeywordTok{sapply}\NormalTok{(predicted.classes.test, }\ControlFlowTok{function}\NormalTok{(x) }\KeywordTok{round}\NormalTok{(x))}
    
\NormalTok{    confussionMatrixTest <-}\StringTok{ }\KeywordTok{table}\NormalTok{(predicted.classes.test, test}\OperatorTok{$}\NormalTok{diabetes)}
\NormalTok{    linreg.accuracy.test <-}\StringTok{ }\NormalTok{(confussionMatrixTest[}\DecValTok{1}\NormalTok{,}\DecValTok{1}\NormalTok{] }\OperatorTok{+}\StringTok{ }\NormalTok{confussionMatrixTest[}\DecValTok{2}\NormalTok{,}\DecValTok{2}\NormalTok{]) }\OperatorTok{/}\StringTok{ }\KeywordTok{length}\NormalTok{(predicted.classes.test)}
\NormalTok{    linreg.precision.test <-}\StringTok{ }\NormalTok{confussionMatrixTest[}\DecValTok{1}\NormalTok{,}\DecValTok{1}\NormalTok{]}\OperatorTok{/}\KeywordTok{sum}\NormalTok{(confussionMatrixTest[,}\DecValTok{1}\NormalTok{])}
\NormalTok{    linreg.recall.test <-}\StringTok{ }\NormalTok{confussionMatrixTest[}\DecValTok{1}\NormalTok{,}\DecValTok{1}\NormalTok{]}\OperatorTok{/}\KeywordTok{sum}\NormalTok{(confussionMatrixTest[}\DecValTok{1}\NormalTok{,])}
\end{Highlighting}
\end{Shaded}

\#\#\#\#\#Accuracy of the model on the train dataset is 0.7654941.

\#\#\#\#\#Precision of the model on the train dataset is 0.8909091.

\#\#\#\#\#Recall of the model on the train dataset is 0.7777778.

\#\#\#\#\#Accuracy of the model on the test dataset is 0.7777778.

\#\#\#\#\#Precision of the model on the test dataset is 0.826087.

\#\#\#\#\#Recall of the model on the test dataset is 0.840708.

\#\#\#Decision boundaries of Linear Regression with different variables

\begin{Shaded}
\begin{Highlighting}[]
\NormalTok{plot.linreg.boundaries <-}\StringTok{ }\ControlFlowTok{function}\NormalTok{(indexA, indexB) \{}
\NormalTok{  grid.A <-}\StringTok{ }\KeywordTok{seq}\NormalTok{(}\KeywordTok{min}\NormalTok{(data[,indexA]), }\KeywordTok{max}\NormalTok{(data[,indexA]), (}\KeywordTok{max}\NormalTok{(data[,indexA]) }\OperatorTok{-}\StringTok{ }\KeywordTok{min}\NormalTok{(data[,indexA])) }\OperatorTok{/}\StringTok{ }\DecValTok{100}\NormalTok{)}
\NormalTok{  grid.B <-}\StringTok{ }\KeywordTok{seq}\NormalTok{(}\KeywordTok{min}\NormalTok{(data[,indexB]), }\KeywordTok{max}\NormalTok{(data[,indexB]), (}\KeywordTok{max}\NormalTok{(data[,indexB]) }\OperatorTok{-}\StringTok{ }\KeywordTok{min}\NormalTok{(data[,indexB])) }\OperatorTok{/}\StringTok{ }\DecValTok{100}\NormalTok{)}
\NormalTok{  grid <-}\StringTok{ }\KeywordTok{expand.grid}\NormalTok{(grid.A,grid.B)}
  \KeywordTok{colnames}\NormalTok{(grid) <-}\StringTok{ }\KeywordTok{colnames}\NormalTok{(trainDf[, }\KeywordTok{c}\NormalTok{(indexA, indexB)])}
  
\NormalTok{  ml <-}\StringTok{ }\KeywordTok{lm}\NormalTok{(}\KeywordTok{as.numeric}\NormalTok{(diabetes)}\OperatorTok{~}\NormalTok{.,}\DataTypeTok{data =}\NormalTok{ train[, }\KeywordTok{c}\NormalTok{(indexA, indexB, }\DecValTok{9}\NormalTok{)])}
\NormalTok{  predicted.classes <-}\StringTok{ }\KeywordTok{round}\NormalTok{(}\KeywordTok{predict}\NormalTok{(ml, grid, }\DataTypeTok{type =} \StringTok{"response"}\NormalTok{))}
  
  \KeywordTok{plot}\NormalTok{(data[, indexA], data[ ,indexB], }\DataTypeTok{pch=}\DecValTok{20}\NormalTok{, }\DataTypeTok{col=}\NormalTok{col1[}\KeywordTok{as.numeric}\NormalTok{(data}\OperatorTok{$}\NormalTok{diabetes)], }\DataTypeTok{xlab=}\KeywordTok{colnames}\NormalTok{(data)[indexA], }\DataTypeTok{ylab=}\KeywordTok{colnames}\NormalTok{(data)[indexB])}
  
  \KeywordTok{points}\NormalTok{(grid[, }\DecValTok{1}\NormalTok{], grid[,}\DecValTok{2}\NormalTok{], }\DataTypeTok{pch=}\StringTok{'.'}\NormalTok{, }\DataTypeTok{col=}\NormalTok{col1[}\KeywordTok{as.numeric}\NormalTok{(predicted.classes)])  }\CommentTok{# draw grid}
  \KeywordTok{legend}\NormalTok{(}\StringTok{"topleft"}\NormalTok{, }\DataTypeTok{legend=}\KeywordTok{levels}\NormalTok{(data}\OperatorTok{$}\NormalTok{diabetes),}\DataTypeTok{fill =}\NormalTok{col1)}
  \KeywordTok{title}\NormalTok{(}\KeywordTok{c}\NormalTok{(}\StringTok{"Classification with linear regression"}\NormalTok{))}
\NormalTok{  predicted.matrix <-}\StringTok{ }\KeywordTok{matrix}\NormalTok{(predicted.classes, }\KeywordTok{length}\NormalTok{(grid.A), }\KeywordTok{length}\NormalTok{(grid.B))}
  \KeywordTok{contour}\NormalTok{(grid.A, grid.B, predicted.matrix, }\DataTypeTok{levels=}\KeywordTok{c}\NormalTok{(}\FloatTok{1.5}\NormalTok{), }\DataTypeTok{drawlabels=}\OtherTok{FALSE}\NormalTok{,}\DataTypeTok{add=}\OtherTok{TRUE}\NormalTok{)}
\NormalTok{\}}
\end{Highlighting}
\end{Shaded}

\hypertarget{section-7}{%
\subsubsection{}\label{section-7}}

\hypertarget{glucose-and-age-3}{%
\paragraph{Glucose and age}\label{glucose-and-age-3}}

Plot of the boundaries between columns glucose and age

\begin{Shaded}
\begin{Highlighting}[]
\KeywordTok{plot.linreg.boundaries}\NormalTok{(}\DecValTok{2}\NormalTok{, }\DecValTok{8}\NormalTok{)}
\end{Highlighting}
\end{Shaded}

\includegraphics{assigment1_files/figure-latex/unnamed-chunk-52-1.pdf}
\#\#\#\# Pressure and mass Plot of the boundaries between columns
pressure and mass

\begin{Shaded}
\begin{Highlighting}[]
\KeywordTok{plot.linreg.boundaries}\NormalTok{(}\DecValTok{3}\NormalTok{, }\DecValTok{6}\NormalTok{)}
\end{Highlighting}
\end{Shaded}

\includegraphics{assigment1_files/figure-latex/unnamed-chunk-53-1.pdf}
\#\#\#\# Triceps and insulin Plot of the boundaries between columns
triceps and insulin

\begin{Shaded}
\begin{Highlighting}[]
\KeywordTok{plot.linreg.boundaries}\NormalTok{(}\DecValTok{4}\NormalTok{, }\DecValTok{5}\NormalTok{)}
\end{Highlighting}
\end{Shaded}

\includegraphics{assigment1_files/figure-latex/unnamed-chunk-54-1.pdf}
\#\#\#\# Age and pedigree Plot of the boundaries between columns age and
pedigree

\begin{Shaded}
\begin{Highlighting}[]
\KeywordTok{plot.linreg.boundaries}\NormalTok{(}\DecValTok{8}\NormalTok{, }\DecValTok{7}\NormalTok{)}
\end{Highlighting}
\end{Shaded}

\includegraphics{assigment1_files/figure-latex/unnamed-chunk-55-1.pdf}

\#\#\#Decision boundaries of linear regression in PCA

Finally we draw decision boundaries with the two first components in
PCA.

\begin{Shaded}
\begin{Highlighting}[]
\NormalTok{plot.linreg.pca <-}\StringTok{ }\ControlFlowTok{function}\NormalTok{() \{}
\NormalTok{  pca.values <-}\StringTok{ }\KeywordTok{prcomp}\NormalTok{(trainDf)}\OperatorTok{$}\NormalTok{x}
\NormalTok{  pca.df <-}\StringTok{ }\KeywordTok{as.data.frame.matrix}\NormalTok{(pca.values[,}\KeywordTok{c}\NormalTok{(}\DecValTok{1}\NormalTok{,}\DecValTok{2}\NormalTok{)])}
\NormalTok{  pca.df}\OperatorTok{$}\NormalTok{diabetes <-}\StringTok{ }\NormalTok{train}\OperatorTok{$}\NormalTok{diabetes}
  
\NormalTok{  grid.PC1 <-}\StringTok{ }\KeywordTok{seq}\NormalTok{(}\KeywordTok{min}\NormalTok{(pca.values[,}\DecValTok{1}\NormalTok{]), }\KeywordTok{max}\NormalTok{(pca.values[,}\DecValTok{1}\NormalTok{]), }\DecValTok{5}\NormalTok{)}
\NormalTok{  grid.PC2 <-}\StringTok{ }\KeywordTok{seq}\NormalTok{(}\KeywordTok{min}\NormalTok{(pca.values[,}\DecValTok{2}\NormalTok{]), }\KeywordTok{max}\NormalTok{(pca.values[,}\DecValTok{2}\NormalTok{]), }\DecValTok{1}\NormalTok{)}
\NormalTok{  grid <-}\StringTok{ }\KeywordTok{expand.grid}\NormalTok{(grid.PC1,grid.PC2)}
  \KeywordTok{colnames}\NormalTok{(grid) <-}\StringTok{ }\KeywordTok{c}\NormalTok{(}\StringTok{'PC1'}\NormalTok{,}\StringTok{'PC2'}\NormalTok{)}
    
\NormalTok{  ml <-}\StringTok{ }\KeywordTok{lm}\NormalTok{(}\KeywordTok{as.numeric}\NormalTok{(diabetes)}\OperatorTok{~}\NormalTok{.,}\DataTypeTok{data=}\NormalTok{pca.df)}
\NormalTok{  predicted.classes <-}\StringTok{ }\KeywordTok{round}\NormalTok{(}\KeywordTok{predict}\NormalTok{(ml, grid, }\DataTypeTok{type =} \StringTok{"response"}\NormalTok{))}
  \KeywordTok{plot}\NormalTok{(pca.values[,}\DecValTok{1}\NormalTok{], pca.values[,}\DecValTok{2}\NormalTok{], }\DataTypeTok{pch=}\DecValTok{20}\NormalTok{, }\DataTypeTok{col=}\NormalTok{col1[}\KeywordTok{as.numeric}\NormalTok{(train}\OperatorTok{$}\NormalTok{diabetes)], }\DataTypeTok{xlab=}\StringTok{'PC1'}\NormalTok{, }\DataTypeTok{ylab=}\StringTok{'PC2'}\NormalTok{)}
  \KeywordTok{points}\NormalTok{(grid}\OperatorTok{$}\NormalTok{PC1, grid}\OperatorTok{$}\NormalTok{PC2, }\DataTypeTok{pch=}\StringTok{'.'}\NormalTok{, }\DataTypeTok{col=}\NormalTok{col1[}\KeywordTok{as.numeric}\NormalTok{(predicted.classes)])  }\CommentTok{# draw grid}
  \KeywordTok{legend}\NormalTok{(}\StringTok{"topleft"}\NormalTok{, }\DataTypeTok{legend=}\KeywordTok{levels}\NormalTok{(data}\OperatorTok{$}\NormalTok{diabetes),}\DataTypeTok{fill =}\NormalTok{col1)}
  \KeywordTok{title}\NormalTok{(}\KeywordTok{c}\NormalTok{(}\StringTok{"PCA classification of linear regression"}\NormalTok{))}
  
\NormalTok{  predicted.matrix <-}\StringTok{ }\KeywordTok{matrix}\NormalTok{(}\KeywordTok{as.numeric}\NormalTok{(predicted.classes), }\KeywordTok{length}\NormalTok{(grid.PC1), }\KeywordTok{length}\NormalTok{(grid.PC2))}
  \KeywordTok{contour}\NormalTok{(grid.PC1, grid.PC2, predicted.matrix, }\DataTypeTok{levels=}\KeywordTok{c}\NormalTok{(}\FloatTok{1.5}\NormalTok{), }\DataTypeTok{drawlabels=}\OtherTok{FALSE}\NormalTok{,}\DataTypeTok{add=}\OtherTok{TRUE}\NormalTok{)}
\NormalTok{\}}
\KeywordTok{plot.linreg.pca}\NormalTok{()}
\end{Highlighting}
\end{Shaded}

\includegraphics{assigment1_files/figure-latex/unnamed-chunk-56-1.pdf}

\#\#Conclusion TODO

\end{document}
